\section{Online Transaction Processing (OLTP)} \label{section:oltp}

\textbf{TODO}: La introducci\'on de esta secci\'on es presentar el concepto de OLTP

Claro, la tecnología de procesamiento de transacciones en línea (OLTP) es un sistema crítico en 
la infraestructura tecnológica de muchas organizaciones, ya que permite el procesamiento de 
transacciones en tiempo real. Esto significa que los datos se actualizan al instante y están 
disponibles para su uso inmediato. Los sistemas OLTP se utilizan en una amplia variedad de 
aplicaciones, desde la banca y las finanzas hasta el comercio electrónico y la atención médica.

\subsection{Objetivos de los sistemas OLTP}
\subsection{Arquitectura de los sistemas OLTP}

los sistemas OLTP se basan en una arquitectura cliente-servidor, donde los clientes envían 
solicitudes al servidor para realizar transacciones en la base de datos. El servidor procesa 
estas solicitudes y actualiza la base de datos en tiempo real, lo que permite una respuesta 
rápida a las solicitudes de los clientes. Los sistemas OLTP también utilizan técnicas de 
concurrencia y control de transacciones para garantizar la integridad de los datos y prevenir 
problemas como la corrupción de datos o la pérdida de información.

\subsection{Modelo Relacional}
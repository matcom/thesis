\section{Online Transaction Processing (OLTP)} \label{section:oltp}

En el ámbito de los sistemas de administración de bases de datos, el Procesamiento de Transacciones en Línea (OLTP) es un 
tipo de procesamiento de datos que consiste en ejecutar una serie de transacciones que ocurren simultáneamente y en 
tiempo real. Los sistemas OLTP están diseñados para garantizar la integridad y coherencia de los datos en un entorno de 
múltiples usuarios.

Los datos transaccionales son información que rastrea las interacciones relacionadas con las actividades de una 
organización. Estas interacciones pueden ser transacciones comerciales, como pagos de clientes, pagos realizados 
a proveedores, movimientos en el inventario de una organización, pedidos recibidos o servicios entregados. Los eventos 
transaccionales, usualmente contienen una dimensión de tiempo, algunos valores 
numéricos y referencias a otros datos\cite{oltpAzure}. Aunque, con el paso de los años, especialmente desde la llegada de 
Internet, la definición de transacción 
se ha expandido para poder abarcar todas las posibles formas de interacción digital entre un usuario y un negocio a través 
de cualquier sensor conectado a la web. Además, incluye cualquier tipo de acción, como descargar pdfs en una 
página web, ver un video específico, entre otros\cite{oltpOracle}.

Los sistemas OLTP se centran principalmente en respaldar las operaciones comerciales diarias. Las transacciones que procesan 
son brebes y sencillas, que normalmente implican la inserción, modificación o recuperación de cantidades pequeñas de datos.

\subsection{Objetivos de los sistemas OLTP}

El objetivo fundamental de los sistemas OLTP es brindar rapidez en las consultas de los usuarios y 
garantizar que los datos permanezcan consistentes y actualizados. De forma m\'as espec\'ifica, los objetivos de 
los sistemas OLTP son: 

\begin{itemize}
    \item Posibilitar el procesamiento de transacciones en tiempo real. La rapidez que brinda OLTP en el procesamiento de 
        las transacciones, permite a las organizaciones mantener datos actualizados y 
        confiables.

    \item Mantener la integridad y la coherencia de los datos durante todo el proceso transaccional. Los sistemas OLTP 
        cumplen las propiedades ACID (Atomicidad, Consistencia, Aislamiento, Durabilidad), por tanto garantizan un 
        procesamiento de datos confiable y sin errores.

    \item Recuperar y manipular datos de forma eficiente para respaldar las operaciones transaccionales. Los sistemas OLTP
        est\'an optimizados para responder consultas e implementan mecanismos de indexación que proporcionan rapidez y presici\'on 
        a la hora de recuperar datos. Adem\'as, ofrecen capacidades s\'olidas de manipulaci\'on de datos, lo que permite 
        insertar, eliminar o actualizar registros de forma segura.

    \item Proporcionar alta disponibilidad y escalabilidad. Es necesario que los sistemas OLTP sean capaces de manejar una 
        gran cantidad de usuarios y transacciones concurrentes sin verse afectado su rendimiento.

    \item Mantener la seguridad y la privacidad de los datos es un objetivo fundamental de los sistemas OLTP. Con este fin, 
        estos sistemas implementan medidas de seguridad, como controles de acceso, encriptación y mecanismos de autenticación, 
        para proteger los datos de manipulaciones o accesos no autorizadas.
\end{itemize}

\subsection{Arquitectura de los sistemas OLTP}

los sistemas OLTP se basan en una arquitectura cliente-servidor, donde los clientes envían 
solicitudes al servidor para realizar transacciones en la base de datos. El servidor procesa 
estas solicitudes y actualiza la base de datos en tiempo real, lo que permite una respuesta 
rápida a las solicitudes de los clientes. Los sistemas OLTP también utilizan técnicas de 
concurrencia y control de transacciones para garantizar la integridad de los datos y prevenir 
problemas como la corrupción de datos o la pérdida de información.

\subsection{Modelo Relacional}
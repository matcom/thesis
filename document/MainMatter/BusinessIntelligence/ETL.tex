\section{Procesos ETL}\label{section:etl}

ETL son las siglas de  Extract, Transform, Load, en español Extraer, Transformar y Cargar. Es un proceso fundamental en la 
integración y gestión de datos. Implica extraer datos desde varios or\'igenes, generalmente con formatos distintos, 
conciliarlos, mediante tecnicas de transformación y validaci\'on, en un formato compatible con un almacenamiento destino 
predefinido y luego cargarlos en dicho sistema destino. Lo que permite aumentar la cantidad y la calidad de los datos 
disponibles para los an\'alisis adem\'as de permitir la integración con sistemas heredados.

\subsection{Objetivos de los procesos ETL}

El objetivo fundamental de los procesos ETL es garantizar la calidad, consistencia y confiabilidad de los datos para 
fines anal\'iticos y de toma de decisiones. Adem\'as, cumple con otros objetivos clave. En una primera instancia, 
tienen el objetivo de consolidar datos de m\'ultiples fuentes, d\'igase bases de datos, hojas de c\'alculo, APIs, 
archivos planos y sistemas externos, en un formato unificado y estandarizado. La consolidación de los datos facilita 
los procesos de an\'alisis y de generación de informes de datos. En segundo lugar, con los procesos ETL se busca 
limpiar y transformar los datos, asegurando que sean precisos, completos y cumplan con las reglas y requisitos comerciales. 
Por \'ultimo, ETL permite la integración con datos en tiempo real e históricos, lo cual brinda a las organizaciones una visión 
completa de sus datos a lo largo del tiempo, mejorando la obtenci\'on de conocimiento y la toma de decisiones.

\subsection{ETL vs ELT}
\subsection{Operaciones de los Procesos ETL}
\subsection{Herramientas para Procesos ETL}

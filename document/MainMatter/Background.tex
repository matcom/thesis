\chapter{Estado del Arte}\label{chapter:state-of-the-art}

El Minado de Argumentos es una rama del Procesamiento del Lenguaje Natural encargado 
de la extracción y anotación de estructuras argumentativas en textos de diferente índole.
Un argumento es una compleja estructura compuesta por una afirmación y un conjunto de proposiciones
que apoyan o atacan al propio argumento o a otras proposiciones CITE (arg def). Dada la definición
anterior se puede observar los problemas fundamentales a resolver:

\begin{itemize}
    \item Extracción de las proposiciones
    \item Clasificación de las proposiciones
    \item Identificar las relaciones entre proposiciones
    \item Clasificar las relaciones entre proposiciones
\end{itemize}

Para las proposiciones existen varios tipos de clasificaciones, se han usado Afirmación mayor, Afirmación y 
Premisa \cite{eger2017neural}, \cite{stab2017parsing}, otros no dependen de clases fijas y se ajustan 
a la anotación del corpus con que se trabaja \cite{galassi2018argumentative}. El modelado de las relaciones
tiene dos vertientes principales. Por una parte están las que restringen estas relaciones a tener una
estructura arbórea \cite{eger2017neural} \cite{stab2017parsing}, lo cual facilita su modelación y 
comprensión, mientras que otras permiten una representación más libre y por lo tanto más compleja 
\cite{galassi2018argumentative} \cite{niculae2017argument}.

En las aproximaciones presentadas para atacar dicho problema se destacan los algoritmos divididos
por etapas en la que cada subproblema es independientemente resuelto y su salida es la entrada del
próximo subproblema \cite{stab2017parsing} \cite{goudas2015argument}, esta aproximación es problemática debido a la propagación del error y en que no 
explotan las interrelaciones entre las variables \cite{eger2017neural}. Otros tipos de algoritmos son
los llamados end-to-end los cuales se enfrentan a los problemas presentados anteriormente de manera
conjunta, partiendo de los tokens de los textos haciendo el procesamiento de tal forma que no se pierda
información y que esta se pueda compartir entre etapas \cite{eger2017neural}.

El estado del arte actualmente en este campo se ha logrado gracias al uso de técnicas de aprendizaje de
máquina. En este ámbito se han creado modelos basados en Support Vector Machine, Recurrent Neural Networks, 
Residual Neural Networks, Attention Neural Networks. Se ha modelado el problema de extracción de argumentos
como problemas de inferencia en un factor graph \cite{niculae2017argument}, parseo de dependencia, etiquetado de
secuencia, extracción de entidades nombradas y sus relaciones \cite{eger2017neural}.

El modelo más promisorio se basa en redes neuronales con módulo de atención y redes residuales. Lo interesante
es que el uso de features dependientes del problema es nula, sus features dependen solo de los argumentos en sí
usando embeddings de BERT. Los resultados de este modelo superan el estado del arte que existía en su actualidad
y deja un camino abierto al perfeccionamiento de este mediante la inclusión de features dependientes de contextos
específicos.

Recomendaciones sobre el contenido
\begin{itemize}
    \item Tener referencias a trabajos relevantes
    \item Presentar marco teórico
    \item Fundamentación teórica
    \item Herramientas a utilizar
    \begin{itemize}
        \item Justificar el uso de la herramienta, poner ejemplos comparativos con otras herramientas
    \end{itemize}
\end{itemize}

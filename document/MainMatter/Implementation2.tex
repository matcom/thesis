\chapter{Detalles de Implementación y Experimentos}\label{chapter:implementation}

% Explicación de las métricas: F1, %F1, Precision, Recobrado
% Con que software se hizo el modelo: Keras, Tensorflow
% Explicacion de corpus estudiados: Decir que los esquemas de argumentacion presentes son diferentes, por lo tanto
% usaron varios para ver cual era el mejor que se acomodaba al conjunto de validación final (Granma)
% Experimentación con segmenter
% Experimentación con link prediction
% Conclusiones: Mejor clasificar y segmentar UDA con segmenter y dejar la clasificacion de UDA como tarea auxiliar en 
% link prediction, poner los mejores ejemplos y peores ejemplos para los mejores modelos.

Para el entrenamiento de los modelos propuestos se utilizaron corpus diferentes, estos
presentas esquemas de anotaci'on distintos entre s'i. Se observ'o el el desempenno
de los modelos con los corpus en las cartas a la direcci'on extra'idas del peri'odico Granma.

\section{Conjuntos de Datos}

\subsection{Ensayos Argumentativos}\label{corpus:persuasive_essays}

\subsection{CDCP}

\subsection{AbsTRCT}

\subsection{DrInventor}

\subsection{Cartas a la dirrecci'on}


\section{Selecci'on de hiperpar'ametros}

Se tom'o el corpus de Ensayos Argumentativos \ref{corpus:persuasive_essays} para la selecci'on 
de hiperpar'ametros y selecci'on del modelo final.  
\documentclass[12pt,oneside]{uhthesis}
\usepackage{subfigure}
\usepackage[ruled,lined,linesnumbered,titlenumbered,algochapter,spanish,onelanguage]{algorithm2e}
\usepackage{amsmath}
\usepackage{amssymb}
\usepackage{amsbsy}
\usepackage{caption,booktabs}
\captionsetup{ justification = centering }
%\usepackage{mathpazo}
\usepackage{float}
\setlength{\marginparwidth}{2cm}
\usepackage{todonotes}
\usepackage{listings}
\usepackage{xcolor}
\usepackage{multicol}
\usepackage{graphicx}
\floatstyle{plaintop}
\restylefloat{table}
\addbibresource{Bibliography.bib}
% \setlength{\parskip}{\baselineskip}%
\renewcommand{\tablename}{Tabla}
\renewcommand{\listalgorithmcfname}{Índice de Algoritmos}
%\dontprintsemicolon
\SetAlgoNoEnd

\definecolor{codegreen}{rgb}{0,0.6,0}
\definecolor{codegray}{rgb}{0.5,0.5,0.5}
\definecolor{codepurple}{rgb}{0.58,0,0.82}
\definecolor{backcolour}{rgb}{0.95,0.95,0.92}

\lstdefinestyle{mystyle}{
    backgroundcolor=\color{backcolour},   
    commentstyle=\color{codegreen},
    keywordstyle=\color{purple},
    numberstyle=\tiny\color{codegray},
    stringstyle=\color{codepurple},
    basicstyle=\ttfamily\footnotesize,
    breakatwhitespace=false,         
    breaklines=true,                 
    captionpos=b,                    
    keepspaces=true,                 
    numbers=left,                    
    numbersep=5pt,                  
    showspaces=false,                
    showstringspaces=false,
    showtabs=false,                  
    tabsize=4
}

\lstset{style=mystyle}

\title{Título de la tesis}
\author{\\\vspace{0.25cm}Nombre del autor}
\advisor{\\\vspace{0.25cm}Nombre del primer tutor\\\vspace{0.2cm}Nombre del segundo tutor}
\degree{Licenciado en (Matemática o Ciencia de la Computación)}
\faculty{Facultad de Matemática y Computación}
\date{Fecha\\\vspace{0.25cm}\href{https://github.com/username/repo}{github.com/username/repo}}
\logo{Graphics/uhlogo}
\makenomenclature

\renewcommand{\vec}[1]{\boldsymbol{#1}}
\newcommand{\diff}[1]{\ensuremath{\mathrm{d}#1}}
\newcommand{\me}[1]{\mathrm{e}^{#1}}
\newcommand{\pf}{\mathfrak{p}}
\newcommand{\qf}{\mathfrak{q}}
%\newcommand{\kf}{\mathfrak{k}}
\newcommand{\kt}{\mathtt{k}}
\newcommand{\mf}{\mathfrak{m}}
\newcommand{\hf}{\mathfrak{h}}
\newcommand{\fac}{\mathrm{fac}}
\newcommand{\maxx}[1]{\max\left\{ #1 \right\} }
\newcommand{\minn}[1]{\min\left\{ #1 \right\} }
\newcommand{\lldpcf}{1.25}
\newcommand{\nnorm}[1]{\left\lvert #1 \right\rvert }
\renewcommand{\lstlistingname}{Ejemplo de código}
\renewcommand{\lstlistlistingname}{Ejemplos de código}

\begin{document}

\frontmatter
\maketitle

\begin{dedication}
    Dedicación
\end{dedication}
\begin{acknowledgements}
    Agradecimientos
\end{acknowledgements}
\begin{opinion}
La lingüística computacional es una rama multidisciplinaria que procesa grandes 
cantidades de textos escritos. La tesis presentada por Luis Ernesto Ibarra Vázquez 
propone un algoritmo basado en modelos de aprendizaje automático para la segmentación 
y clasificación de enunciados argumentativos en textos de la sección “Cartas a la Dirección” 
del periódico \emph{Granma}.

La tesis se ubica en el marco de un proyecto \emph{Dinámicas sociales, políticas y económicas en el 
discurso público en Cuba de principio del siglo XXI: estudios de CORESPUC}, asociado al Programa 
Nacional de Ciencia y Técnica “Las Ciencias Sociales y las humanidades. Desafíos ante la estrategia 
de desarrollo de la sociedad cubana”, Código PN223LH011-011, Ministerio de Ciencia, Tecnología y 
Medio Ambiente (CITMA), Cuba, 2021-2023.

Por ello, Luis Ernesto tuvo que estudiar la materia referida, que no está incluida en el currículo 
de la carrera y trabajó mostrando creatividad, disciplina, entrega y rigor. Deseo destacar su 
profundización en el estudio computacional de la argumentación y la proposición de soluciones de 
software útiles para el análisis de enunciados argumentativos en español.

La investigación realizada por el estudiante incluyó diversos corpus en idioma inglés con 
diferentes anotaciones de la argumentación que se usaron para el entrenamiento de modelos de 
aprendizaje automático proyectivos para clasificar los enunciados argumentativos en español, a 
partir del conocimiento de la anotación en idioma inglés. Se compararon varios modelos y el mejor 
fue utilizado para la clasificación de enunciados de las Cartas a la Dirección del periódico Granma. 
Este trabajo es un primer paso positivo en el estudio automático de la argumentación de nuestra variante 
del español y llevará una posterior revisión por lingüistas. Dicho corpus anotado será importante para la 
realización de análisis sociolingüísticos del corpus donde la argumentación es una variable de interés para 
el proyecto CORESPUC.

Durante el desarrollo del trabajo, Luis Ernesto demostró habilidades para el trabajo con la bibliografía y 
creatividad para proponer soluciones a problemas de implementación, entre otras competencias de programación 
en el lenguaje Python y sus diversos \emph{frameworks}. Así, se logró cumplir el objetivo de esta tesis.

Por tanto, considero que al estudiante Luis Ernesto Ibarra Vázquez debe otorgársele la máxima calificación 
(5 puntos, Excelente), y estoy seguro que en el futuro Luis Ernesto se desempeñará como un excelente profesional 
de la Ciencia de la Computación.

\begin{flushright}
\includegraphics[scale=0.2]{Graphics/firma.jpg}\\
MSc. Damian Valdés Santiago\\
21 de noviembre de 2022    
\end{flushright}

\end{opinion}
\begin{resumen}
	Resumen en español
\end{resumen}

\begin{abstract}
	Resumen en inglés
\end{abstract}
\tableofcontents
\listoffigures
% \listoftables
% \listofalgorithms
\lstlistoflistings

\mainmatter

\chapter*{Introducción}\label{chapter:introduction}
\addcontentsline{toc}{chapter}{Introducción}

% Introducción a la teoría de la argumentación

La teoría de la argumentación es el estudio interdisciplinario de cómo las conclusiones
pueden ser apoyadas o socavadas por premisas a través de razonamiento lógico [\cite{wiki-arg-theory}] .
Esta es usada en varios aspectos de la vida como en las negociaciones, debates públicos, publicaciones
científicas, enseñanza, leyes. Sobre este marco teórico es sobre donde se realiza el diseño y análisis de 
modelos computacionales que ayudan al procesamiento de los grandes volúmenes de datos existentes.

En la actualidad es necesario tener acceso a la información necesaria
de forma rápida y simple. Esto no siempre es posible dado la gran cantidad de información existente y
que es generada en cada momento. En caso de tener una vía de acceder a esta se podrían realizar acciones
con mayor rapidez y calidad. Con la argumentación se podría hacer explícitas las razones de las personas 
al afirmar algo sobre un tema teniendo así su punto de vista individual, y con suficientes personas, colectivo.

% Presentación de la problemática

La Extracción de Argumentos (EA) es la rama del Procesamiento de Lenguaje Natural encargada de formularizar
y modelar dicho problema. Los modelos existentes generalmente se encargan de
extraer y clasificar las componentes argumentativas y sus relaciones de una fuente no estructurada de 
texto, capturando un proceso de razonamiento y argumentación en su estructura final. Los métodos
de realizar este procedimiento varían en dependencia de las características en que se trabaja y el objetivo
a que se quiere llegar. Entre los métodos usados se pueden mencionar métodos ad-hoc basados en gramáticas,
que aprovechan atributos del texto como partes de la oración y entidades nombradas, que codifican diferentes 
patrones argumentativos [\cite{dykes2020reconstructing}]. Dicho enfoque requiere de trabajo 
humano para la creación de una gramática que permita resultados satisfactorios. Este tipo de enfoque
generalmente sacrifica recobrado por precisión y no permite una generalización del problema, contribuyendo
a no ser muy escalable.

Existen método basados en técnicas de aprendizaje de máquina para modelar y resolver dicho problema.
Estos métodos están divididos en diferentes vertientes de acuerdo a cómo procesan los datos. Uno de ellos lo realiza por
etapas, en el cual en cada una se resuelve independientemente los subproblema y la salida de la etapa 
anterior es entrada a la etapa siguiente. Las etapas en que generalmente se divide el problema
son, separación de unidades argumentativas y no argumentativas, clasificación de las unidades 
argumentativas y la identificación de estructuras argumentativas [\cite{stab2014identifying}].
Este enfoque trae consigo una modularidad elevada al resolver las tareas de manera independiente, pero
tiene la desventaja que los errores de etapas anteriores son pasados a las siguientes y además los algoritmos
usados no ven todo el contexto del texto pudiendo perder características que permitirían un mejor resultado.
Otro enfoque usado son los llamados end-to-end, en estos el modelo entrenado aprende los pasos para convertir
directamente la entrada del algoritmo en la salida deseada al entrenar sus diferentes partes de manera 
simultanea. En el contexto de EA la entrada serían los tokens de los textos y la salida sería las
estructuras argumentativas anotadas en dependencia del algoritmo usado. Este enfoque mitiga las posibles
deficiencias del enfoque por etapas, al juntar todo el proceso en una sola eliminando la propagación
del error, además de que al tener todos los datos es posible encontrar mayor cantidad de correlaciones 
entre ellos, también no requiere de una ingeniería de atributos tan elaborada [\cite{eger2017neural}].

En la práctica la EA tiene un gran número de aplicaciones [\cite{janier2019argument}]: 
\begin{itemize}
    \item Análisis de opinión: Ayuda no solo a saber si la opinión es favorable o no, sino a saber
    porqué es favorable o no.
    \item Análisis de debate: Ayuda a detectar estrategias argumentativas
    \item Detección de incoherencias en un conjunto de argumentos y justificaciones
\end{itemize}

% Actualidad, novedad e importancia

Desde el punto de vista del proyecto CORESPUC, este trabajo añade la capacidad de anotar
sus textos con las estructuras argumentativas pertinentes ampliando cantidad de
información anotada en este. Desde el punto de vista de Extracción de Argumentos, supone una
adición al estudio de este campo en el lenguaje español, del cual se encontraron pocas
investigaciones realizadas [\cite{esteve2020mineria}]. Con respecto a Cuba en específico
resulta un trabajo completamente nuevo según las investigaciones hechas.

% Pregunta científica o hipótesis

Para la solución del problema se necesita extraer las componentes argumentativas con sus relaciones
y clasificaciones. Para esto se necesita encontrar qué modelo es factible usar para la extracción de
argumentos en textos en español, especialmente en la prensa. En este aspecto existen investigaciones
en las cuales usan modelos basados en Transformer y Attention que han impuesto nuevos estados de arte
[\cite{mayer2020transformer}, \cite{galassi2018argumentative}].

% Este trabajo necesita encontrar \emph{qué modelos se pueden usar en el español para la extracción de 
% argumentos en textos, especialmente en la prensa} (TODO Posible Pregunta Científica?). Recientemente se ha introducido modelos basados 
% en Transformer y Attention (CITE \cite{mayer2020transformer}, \cite{galassi2018argumentative})
% que han llegado a alcanzar resultados iguales o superiores al estado del arte de su momento (TODO Posible Hipótesis?). 
% Por otro lado dado que los corpus están en inglés se desea saber si \emph{es factible usar métodos
% para poder usar el conocimiento aprendido de los algoritmos entrenados en inglés en el español} 
% (TODO Posible Pregunta Científica).

% Objetivos 

El objetivo principal de este trabajo fue el diseño e implementación de un algoritmo para 
el estudio de la argumentación en el periódico digital Granma. Para esto primero
fue necesaria la construcción de un corpus sobre el periódico. Se recolectaron los documentos
del periódico mediante técnicas de scrapping para conformar un corpus inicial no anotado. Este
fue anotado con las estructuras argumentativas por el modelo previamente entrenado usando técnicas 
de proyección entre lenguajes [\cite{eger2018cross}]. Además de las anotaciones anteriores, se agregarán 
otras, como partes de la oración y entidades nombradas para finalmente añadir el corpus a CORESPUC.

% Estructura del trabajo

El trabajo está conformado por \dots (TODO Poner la estructura del trabajo)


% Esqueleto
% \begin{itemize}
%     \item Hablar sobre el conocimiento y el pensamiento del ser humano como ser racional.
%     No existe una verdad única, si no diferente tipos de verdades para diferentes grupos de personas.
%     Cada grupo de personas presentan argumentos por los cuales creen esas verdades y no creen otras.
%     \item Introducir el tema de la argumentación en el NLP, su usos actuales e importancia.
%     \item Introducción de la problemática (Formar un corpus de Granma con estructuras argumentativas),
%     el porqué se quiere hacer esto (Justificación del proyecto CORESPUC, crear un estudio en español del tema)
%     \item Hablar sobre la importancia teórica y práctica del trabajo. No existen estudios en español,
%     los corpus en español son escasos. Ayuda a resolver la justificación de CORESPUC
%     \item Planteo de los objetivos y las preguntas científicas (Crear corpus argumentativo en Español y un framework para la extracción de argumentos en periódicos)
%     \item Estructura del trabajo
% \end{itemize}


% Aspectos que debe tratar la introducción (Se deben de decir implícitamente en los párrafos):

% \begin{itemize}

%     \item Contexto histórico-social donde se desarrolla
%     \item Antecedentes del problema, justificación y motivación. Cómo se ha estudiado primero a mano y luego computacionalmente el problema en la prensa. Motivacion, el proyecto esta integrado en un proyecto nacional  reconocido CORESPUC, lo cual tiene una justificación también.
%     \item Breve presentación de la problemática. (No es el estado del arte aunque se puede hablar un poco de él) Elementos involucrados en el punto de vista cientifico, lleva corpus.
%     \item Actualidad, novedad e importancia teórica y práctica. Revisar literatura (Actualidad, en español no tiene mucho estudio), Se propone un modelo computacional para estudiar ese asutnto que se han propuesto poco, para Cuba no hay ninguno y poco de investigación. 
%     \item Diseño teórico.

%     \begin{itemize}

%         \item Problema: 
%         \item Objeto de Investigación: Procesamiento de Lenguaje Natural
%         \item Campo de acción: Linguistica computacional
%         \item Hipótesis o preguntas científicas
%         \item Objetivos generales y específicos

%         \begin{itemize}
%             \item General
%             \begin{itemize}
%                 \item Diseño e implementación de un algoritmo para el estudio de la argumentación en el periódico digital Granma
%             \end{itemize}
%             \item Específicos
%             \begin{itemize}
%                 \item Construcción del corpus de los periódicos: Crawler, Anotación (scpaCy).
%                 \item Arreglar etiquetas en corpus activo con archivos VRT.
%                 \item Clasificar las secciones en de opinión o no.
%                 \item Verificar la aparición o no de noticias entre la versión PDF y la versión online del periódico.
%                 \item Implementacion de la interfaz gráfica para consultar los resultados.
%                 \item Lograr interoperailidad de la plataforma CQPweb.
%             \end{itemize}
%         \end{itemize}

%     \end{itemize}

%     \item Estructura del trabajo

% \end{itemize}
\chapter{Argumentación}\label{chapter:argumentation}

En el captítulo se aborda la argumentación, definiciones y marcos teóricos existentes para el estudio de esta.
Luego se introduce la Extracción de Argumentos como campo de la lingüísticas computacional encargado del estudio 
y procesamiento de las estructuras argumentativas de textos. También se definen y explican los componentes y las tareas 
asociadas al problema de realizar la extracción de argumentos.

\section{Argumentación}

La argumentación es un tema tratado desde la antigüedad, Aristóteles lo defendía como la 
habilidad de, dada una pregunta, considerar los elementos útiles para persuadir a alguien, algo
similar a la retórica. De una perspectiva más contemporánea surgen las ideas de 
\textcite{perelman1969rhetoric}
enfocadas en un análisis de la retórica en donde se estipula que la teoría de la argumentación
responde a provocar o aumentar la adhesión de las personas a las tesis presentadas, por medio de 
técnicas discursivas. En 
\textcite{toulmin_2003}
se considera como argumento todo aquello que ofrece, 
o todo lo que es utilizado, para justificar o refutar una proposición. En este último, se toma 
una perspectiva más racional y deductiva de la argumentación, dando como resultado lo que se 
conoce como el Método de Toulmin. 

\subsection{Método de Toulmin}

Este método divide los argumentos en seis partes: afirmación 
(\emph{claim}), fundamento (\emph{grounds}), justificación (\emph{warrant}), calificador 
(\emph{qualifier}), refutación (\emph{rebuttal}) y respaldo (\emph{backing}).
Mediante las afirmaciones se conoce el argumento principal que el autor quiere probar a la audiencia,
estas son respaldadas con fundamentos, siendo estos las evidencias y hechos en que se apoya el autor.
Las justificaciones pueden estar explícitas o implícitas y son suposiciones que vinculan los
fundamentos con las afirmaciones, estas a su vez pueden ser respaldadas por conocimiento.
El esquema introduce la posibilidad de otra situación válida a la establecida en las afirmaciones
mediante la refutación. Los calificadores son usados para dar más información de la calidad o seguridad
de las afirmaciones dadas. Un ejemplo\footnote{Extraído de  
\textcite{toulminArgument}.
} de este esquema es:

\begin{adjustwidth}{25pt}{25pt}
    [\emph{Se escucharon ladridos y aullidos en la distancia}]$_{\mathrm{fundamento}}$, 
    [\emph{probablemente}]$_{\mathrm{calificador}}$ 
    [\emph{haya perros en las cercanías}]$_{\mathrm{afirmación}}$.
\end{adjustwidth}

En este ejemplo, además de las partes explícitas, se encuentran partes implícitas como la justificación 
(\emph{los perros son animales que ladran y aúllan}), el respaldo (\emph{se sabe que existen perros en la zona}) y 
la refutación (\emph{puede ser que hayan lobos o coyotes cerca}).

Este método crea una definición compacta que ayuda a los investigadores a enfocar su búsqueda 
en las diferentes categorías definidas. Además, engloba de manera comprensible un tema tan complejo 
como la argumentación al tomar en cuenta gran parte de los elementos presentes en el razonamiento
realizado para llegar a conclusiones, incorporando incluso elementos probabilísticos en el proceso. 

\subsection{Rasgos lingüísticos}

Los rasgos lingüísticos son aquellas características que se encuentran presentes en los textos 
que hacen que estos se clasifiquen argumentativos [\cite{venegas2005hacia}]. Con 
la identificación de estos se hace la tarea de extracción más sencilla y con un marco teórico 
que respalde las decisiones tomadas. Ejemplos de rasgos presentes en textos argumentativos:

\begin{enumerate}
    \item Marcas de orden que introducen párrafos: \emph{primero}, \emph{segundo}, \emph{por un lado}, 
    \emph{por otra parte}, \emph{finalmente}.
    \item Comillas y citas: citar palabras que refuercen la intervención recurriendo a autoridades
    o personajes.
    \item Nexos que expresan causa o consecuencias: \emph{ya que}, \emph{porque}, \emph{pues}, 
    \emph{con motivo de}, \emph{gracias a}, \emph{considerando que}, \emph{por lo tanto}, \emph{de manera que}.
\end{enumerate}

Estos rasgos además de dar indicación de la existencia de argumentos dan pie para conocer las relaciones
entre estos y los tipos de argumentos. Por ejemplo, \emph{por lo tanto}, implica que lo que viene 
a continuación es una conclusión apoyada en lo dicho anteriormente en el texto. Algo parecido
sucede con \emph{ya que}, en este caso implica que lo siguiente es un argumento que se encuentra 
relacionado con lo mencionado antes.

En
\textcite{venegas2005hacia}
se determinan 16 categorías y 51 rasgos lingüísticos, dando una idea 
de la gran variedad de marcadores presentes en la argumentación.

\section{Extracción de Argumentos}

El PLN es un subcampo de la Inteligencia Artificial que tiene como objetivo la comprensión 
del lenguaje humano por las computadoras. 
Mediante el uso de sus algoritmos es posible el procesamiento masivo de texto para la extracción de información 
relevante de este. Entre las tareas pertenecientes a dicho campo se encuentran la Traducción Automática, 
la Generación de Lenguaje Natural y la Extracción de Argumentos (EA). La EA constituye la identificación y extracción 
automática de las estructuras de inferencia y 
razonamiento expresadas como argumentos presentes en el lenguaje natural [\cite{lawrence2020argument}].
En la actualidad, tareas del PLN como el análisis de sentimientos permiten 
extraer cuáles son las opiniones o sentimientos presentes, sin embargo, este análisis presenta una falta 
de información, ya que no justifica el porqué de las opiniones. La EA permite dar respuesta a este problema presentando
los argumentos y cómo sus relaciones justifican las posiciones del hablante. Dicho problema está constituido por diferentes 
estructuras y se compone de distintas tareas necesarias para su solución.

\subsection{Estructuras Argumentativas}

Las estructuras argumentativas son las partes de la argumentación de los textos y sus relaciones.
Estas se componen de dos elementos principales: las Unidades de Discurso Argumentativas (UDA) y los enlaces
existentes entre estas. Las UDAs corresponden a la unidad mínima de argumentación, definida 
como un segmento de texto que juega un solo rol para el argumento analizado, y es 
delimitado por segmentos vecinos que tienen roles diferentes o ningún rol [\cite{stede2018argumentation}].
Las UDAs se relacionan entre sí conformando el proceso de inferencia y razonamiento del argumento.
Tanto los enlaces como las UDAs son clasificados en dependencia de su rol en la argumentación, estas clasificaciones 
en las UDAs parten de los conceptos de afirmación, declaración controversial y parte central del argumento, y premisa,
razones que la justifican o refutan, y en las relaciones de ataque y apoyo. 

\subsection{Tareas de extracción de argumentos}

Dada la definición de estructuras argumentativas y que el objetivo de la EA es extraerlas,
se conciben las siguientes tareas principales:

\subsubsection{Extracción de UDAs}

Consiste en la separación de los segmentos de texto que formarán parte de la estructura.
En este proceso el texto es segmentado y se obtiene un conjunto de UDAs. En el siguiente 
ejemplo\footnote{Traducido del corpus de 
\textcite{stab2017parsing}
.} se representa 
la extracción de UDAs marcadas en \emph{cursiva} de un texto dado:

\begin{adjustwidth}{25pt}{25pt}
    En primer lugar, [\emph{el correo electrónico puede contar como uno de los resultados
    más beneficiosos de la tecnología moderna}]. [\emph{Años atrás, las personas pagaban gran cantidad de dinero para 
    enviar sus cartas y sus pagos estaban sujetos al peso de sus cartas o paquetes y muchos accidentes podrían 
    causar problemas que causarían que el correo no fuera enviado}].
\end{adjustwidth}

\subsubsection{Clasificación de UDAs}

La clasificación de UDAs consiste en asignarle la categoría que toma la UDA en la argumentación. En general, 
las clasificaciones parten de dos clases bases, las afirmaciones y premisas, aunque estas pueden ser tantas
como sea necesario por el problema específico a tratar. Siguiendo con el ejemplo, se observa la clasificación
en afirmación y premisa asignada a los segmentos extraídos en el paso anterior:

\begin{adjustwidth}{25pt}{25pt}
    En primer lugar, [\emph{el correo electrónico puede contar como uno de los resultados
    más beneficiosos de la tecnología moderna}]$_{\mathrm{Afirmación}}$. [\emph{Años atrás, las personas pagaban gran cantidad de dinero para 
    enviar sus cartas y sus pagos estaban sujetos al peso de sus cartas o paquetes y muchos accidentes podrían 
    causar problemas que causarían que el correo no fuera enviado}]$_{\mathrm{Premisa}}$.
\end{adjustwidth}

\subsubsection{Extracción de relaciones entre las UDAs}

La extracción de relaciones constituye el paso donde se determina si están relacionadas las UDAs o no. 
La disposición de estas relaciones forma el proceso de razonamiento en que se basa el autor para validar 
su posición. En el ejemplo se representa la existencia de relación mediante su distancia argumentativa con 
la UDA con la que se relaciona. La distancia argumentativa es la cantidad de UDAs del texto que separan la 
UDA fuente del objetivo [\cite{galassi2021deep}], en caso de ser negativa (positiva) el objetivo se encuentra 
antes (después) que la fuente.

\begin{adjustwidth}{25pt}{25pt}
    En primer lugar, [\emph{el correo electrónico puede contar como uno de los resultados
    más beneficiosos de la tecnología moderna}]$_{\mathrm{Afirmación}}$. [\emph{Años atrás, las personas pagaban gran cantidad de dinero para 
    enviar sus cartas y sus pagos estaban sujetos al peso de sus cartas o paquetes y muchos accidentes podrían 
    causar problemas que causarían que el correo no fuera enviado}]$_{\mathrm{Premisa, -1}}$.
\end{adjustwidth}

\subsubsection{Clasificación de relaciones entre las UDAs}

La clasificación de las relaciones consiste en clasificar las relaciones extraídas en las categorías pertinentes.
Los tipos de relaciones, nacen de dos clases bases por lo general, las relaciones de apoyo y de ataque.
Las de apoyo son aquellas en las que la UDA fuente afirme la UDA objetivo, las de ataque son 
las que la UDA fuente apoya la negación de la UDA objetivo.

\begin{adjustwidth}{25pt}{25pt}
    En primer lugar, [\emph{el correo electrónico puede contar como uno de los resultados
    más beneficiosos de la tecnología moderna}]$_{\mathrm{Afirmación}}$. [\emph{Años atrás, las personas pagaban gran cantidad de dinero para 
    enviar sus cartas y sus pagos estaban sujetos al peso de sus cartas o paquetes y muchos accidentes podrían 
    causar problemas que causarían que el correo no fuera enviado}]$_{\mathrm{Premisa, -1, apoyo}}$.
\end{adjustwidth}

Partiendo de esto, se puede observar que las estructuras argumentativas de un texto constituyen un grafo dirigido 
en donde sus nodos representan las UDAs y están anotados con su tipo, y sus vértices representan las 
relaciones entre las UDAs. Dichos vértices se anotan con el tipo de relación existente entra ambas 
(Figura \ref{fig:arg_struct}).

\begin{figure}[h!]
	\begin{center}
		\begin{center}
            \includesvg[scale=.7]{Graphics/Estructuras_argumentativas.svg}
        \end{center}
        % TODO Anotación {y sus relaciones} reduntante, la definición de estructuras argumentativa ya las engloba 
	    \caption{Estructuras Argumentativas.}
        \label{fig:arg_struct}
	\end{center}
\end{figure}

\chapter{Propuesta}\label{chapter:proposal}

\begin{itemize}

\item Modelo del fenómeno
\item Diseño conceptual
\begin{itemize}

\item Entrada $\rightarrow$ Procesamiento $\rightarrow$ Salida
\item Cómo se hace cada etapa

\end{itemize}

\end{itemize}
\chapter{Detalles de Implementación y Experimentos}\label{chapter:implementation}


\begin{itemize}

    \item No interesa el cómo se hizo
    \item Mostrar resultados
    \begin{itemize}

        \item Puede ser con estadísticas
        \item Por casos de uso

    \end{itemize}
    \item Sección de metacódigo en caso de ser relevante

\end{itemize}


\backmatter

\begin{conclusions}

% TODO Como se cumplieron los objetivos de la introduccion y para que sirve lo que hice

En el trabajo se implement'o un algoritmo con el cual se hace posible la extracción de las estructuras
argumentativas de textos 


\end{conclusions}

\begin{recomendations}
    Recomendaciones

    \begin{enumerate}
        \item Usar metodos de Graph Neural Networks para modelar las relaciones entre las UDA. (Actualmente las relaciones extraídas son independientes)
        \item Ajustar (Tune) awesome-align con inglés-español.
        \item Usar otros embeddings para la representación de palabras (BERT, SciBERT, RoBERTa).
        \item Desplegar un servidor Brat para socializar los resultados y permitir que los investigadores perfeccionen las anotaciones.
        \item Validación externa con lingüistas que permitan generalizar el uso del modelo.
    \end{enumerate}
\end{recomendations}

\printbibliography[heading=bibintoc]

%\begin{thebibliography}{99}
%
%\bibitem{texbook}
%Donald E. Knuth (1986) \emph{The \TeX{} Book}, Addison-Wesley Professional.
%
%
%\end{thebibliography}
%\begin{document}
%\bibliography{Bibliography}
%
%\end{document}


\end{document}
\documentclass[12pt,oneside]{uhthesis}
\usepackage{subfigure}
\usepackage[ruled,lined,linesnumbered,titlenumbered,algochapter,onelanguage]{algorithm2e}
\usepackage{amsmath}
\usepackage{amssymb}
\usepackage{amsbsy}
\usepackage{caption,booktabs}
\captionsetup{ justification = centering }
%\usepackage{mathpazo}
\usepackage{float}
\setlength{\marginparwidth}{2cm}
\usepackage{todonotes}
\usepackage{listings}
\usepackage{xcolor}
\usepackage{changepage}
\usepackage{multicol}
\usepackage{graphicx}
\usepackage{svg}
\floatstyle{plaintop}
\restylefloat{table}
\addbibresource{Bibliography.bib}
% \setlength{\parskip}{\baselineskip}%
\renewcommand{\tablename}{Tabla}
\renewcommand{\listalgorithmcfname}{Índice de Algoritmos}
%\dontprintsemicolon
\SetAlgoNoEnd
\DeclareUnicodeCharacter{0301}{í}
\definecolor{codegreen}{rgb}{0,0.6,0}
\definecolor{codegray}{rgb}{0.5,0.5,0.5}
\definecolor{codepurple}{rgb}{0.58,0,0.82}
\definecolor{backcolour}{rgb}{0.95,0.95,0.92}

\lstdefinestyle{mystyle}{
    backgroundcolor=\color{backcolour},   
    commentstyle=\color{codegreen},
    keywordstyle=\color{purple},
    numberstyle=\tiny\color{codegray},
    stringstyle=\color{codepurple},
    basicstyle=\ttfamily\footnotesize,
    breakatwhitespace=false,         
    breaklines=true,                 
    captionpos=b,                    
    keepspaces=true,                 
    numbers=left,                    
    numbersep=5pt,                  
    showspaces=false,                
    showstringspaces=false,
    showtabs=false,                  
    tabsize=4
}

\lstset{style=mystyle}

\title{Extracción automática de argumentos en textos de opinión en la prensa cubana}
\author{\\\vspace{0.25cm}Luis Ernesto Ibarra Vázquez}
\advisor{\\\vspace{0.25cm}MSc. Damian Valdés Santiago}
\degree{Licenciado en Ciencia de la Computación}
\faculty{Facultad de Matemática y Computación}
\date{28 de noviembre del 2022\\\vspace{0.25cm}\href{https://github.com/luisoibarra/thesis}{github.com/luisoibarra/thesis}}
\logo{Graphics/uhlogo}
\makenomenclature

\renewcommand{\vec}[1]{\boldsymbol{#1}}
\newcommand{\diff}[1]{\ensuremath{\mathrm{d}#1}}
\newcommand{\me}[1]{\mathrm{e}^{#1}}
\newcommand{\pf}{\mathfrak{p}}
\newcommand{\qf}{\mathfrak{q}}
%\newcommand{\kf}{\mathfrak{k}}
\newcommand{\kt}{\mathtt{k}}
\newcommand{\mf}{\mathfrak{m}}
\newcommand{\hf}{\mathfrak{h}}
\newcommand{\fac}{\mathrm{fac}}
\newcommand{\maxx}[1]{\max\left\{ #1 \right\} }
\newcommand{\minn}[1]{\min\left\{ #1 \right\} }
\newcommand{\lldpcf}{1.25}
\newcommand{\nnorm}[1]{\left\lvert #1 \right\rvert }
\renewcommand{\lstlistingname}{Ejemplo de código}
\renewcommand{\lstlistlistingname}{Ejemplos de código}

\begin{document}

\frontmatter
\maketitle

\begin{dedication}
    Dedicación
\end{dedication}
\begin{acknowledgements}
    Agradecimientos
\end{acknowledgements}
\begin{opinion}
La lingüística computacional es una rama multidisciplinaria que procesa grandes 
cantidades de textos escritos. La tesis presentada por Luis Ernesto Ibarra Vázquez 
propone un algoritmo basado en modelos de aprendizaje automático para la segmentación 
y clasificación de enunciados argumentativos en textos de la sección “Cartas a la Dirección” 
del periódico \emph{Granma}.

La tesis se ubica en el marco de un proyecto \emph{Dinámicas sociales, políticas y económicas en el 
discurso público en Cuba de principio del siglo XXI: estudios de CORESPUC}, asociado al Programa 
Nacional de Ciencia y Técnica “Las Ciencias Sociales y las humanidades. Desafíos ante la estrategia 
de desarrollo de la sociedad cubana”, Código PN223LH011-011, Ministerio de Ciencia, Tecnología y 
Medio Ambiente (CITMA), Cuba, 2021-2023.

Por ello, Luis Ernesto tuvo que estudiar la materia referida, que no está incluida en el currículo 
de la carrera y trabajó mostrando creatividad, disciplina, entrega y rigor. Deseo destacar su 
profundización en el estudio computacional de la argumentación y la proposición de soluciones de 
software útiles para el análisis de enunciados argumentativos en español.

La investigación realizada por el estudiante incluyó diversos corpus en idioma inglés con 
diferentes anotaciones de la argumentación que se usaron para el entrenamiento de modelos de 
aprendizaje automático proyectivos para clasificar los enunciados argumentativos en español, a 
partir del conocimiento de la anotación en idioma inglés. Se compararon varios modelos y el mejor 
fue utilizado para la clasificación de enunciados de las Cartas a la Dirección del periódico Granma. 
Este trabajo es un primer paso positivo en el estudio automático de la argumentación de nuestra variante 
del español y llevará una posterior revisión por lingüistas. Dicho corpus anotado será importante para la 
realización de análisis sociolingüísticos del corpus donde la argumentación es una variable de interés para 
el proyecto CORESPUC.

Durante el desarrollo del trabajo, Luis Ernesto demostró habilidades para el trabajo con la bibliografía y 
creatividad para proponer soluciones a problemas de implementación, entre otras competencias de programación 
en el lenguaje Python y sus diversos \emph{frameworks}. Así, se logró cumplir el objetivo de esta tesis.

Por tanto, considero que al estudiante Luis Ernesto Ibarra Vázquez debe otorgársele la máxima calificación 
(5 puntos, Excelente), y estoy seguro que en el futuro Luis Ernesto se desempeñará como un excelente profesional 
de la Ciencia de la Computación.

\begin{flushright}
\includegraphics[scale=0.2]{Graphics/firma.jpg}\\
MSc. Damian Valdés Santiago\\
21 de noviembre de 2022    
\end{flushright}

\end{opinion}
\begin{resumen}
	Resumen en español
\end{resumen}

\begin{abstract}
	Resumen en inglés
\end{abstract}
\tableofcontents
\listoffigures
% \listoftables
% \listofalgorithms
\lstlistoflistings

\mainmatter

\chapter*{Introducción}\label{chapter:introduction}
\addcontentsline{toc}{chapter}{Introducción}

% Introducción a la teoría de la argumentación

La teoría de la argumentación es el estudio interdisciplinario de cómo las conclusiones
pueden ser apoyadas o socavadas por premisas a través de razonamiento lógico [\cite{wiki-arg-theory}] .
Esta es usada en varios aspectos de la vida como en las negociaciones, debates públicos, publicaciones
científicas, enseñanza, leyes. Sobre este marco teórico es sobre donde se realiza el diseño y análisis de 
modelos computacionales que ayudan al procesamiento de los grandes volúmenes de datos existentes.

En la actualidad es necesario tener acceso a la información necesaria
de forma rápida y simple. Esto no siempre es posible dado la gran cantidad de información existente y
que es generada en cada momento. En caso de tener una vía de acceder a esta se podrían realizar acciones
con mayor rapidez y calidad. Con la argumentación se podría hacer explícitas las razones de las personas 
al afirmar algo sobre un tema teniendo así su punto de vista individual, y con suficientes personas, colectivo.

% Presentación de la problemática

La Extracción de Argumentos (EA) es la rama del Procesamiento de Lenguaje Natural encargada de formularizar
y modelar dicho problema. Los modelos existentes generalmente se encargan de
extraer y clasificar las componentes argumentativas y sus relaciones de una fuente no estructurada de 
texto, capturando un proceso de razonamiento y argumentación en su estructura final. Los métodos
de realizar este procedimiento varían en dependencia de las características en que se trabaja y el objetivo
a que se quiere llegar. Entre los métodos usados se pueden mencionar métodos ad-hoc basados en gramáticas,
que aprovechan atributos del texto como partes de la oración y entidades nombradas, que codifican diferentes 
patrones argumentativos [\cite{dykes2020reconstructing}]. Dicho enfoque requiere de trabajo 
humano para la creación de una gramática que permita resultados satisfactorios. Este tipo de enfoque
generalmente sacrifica recobrado por precisión y no permite una generalización del problema, contribuyendo
a no ser muy escalable.

Existen método basados en técnicas de aprendizaje de máquina para modelar y resolver dicho problema.
Estos métodos están divididos en diferentes vertientes de acuerdo a cómo procesan los datos. Uno de ellos lo realiza por
etapas, en el cual en cada una se resuelve independientemente los subproblema y la salida de la etapa 
anterior es entrada a la etapa siguiente. Las etapas en que generalmente se divide el problema
son, separación de unidades argumentativas y no argumentativas, clasificación de las unidades 
argumentativas y la identificación de estructuras argumentativas [\cite{stab2014identifying}].
Este enfoque trae consigo una modularidad elevada al resolver las tareas de manera independiente, pero
tiene la desventaja que los errores de etapas anteriores son pasados a las siguientes y además los algoritmos
usados no ven todo el contexto del texto pudiendo perder características que permitirían un mejor resultado.
Otro enfoque usado son los llamados end-to-end, en estos el modelo entrenado aprende los pasos para convertir
directamente la entrada del algoritmo en la salida deseada al entrenar sus diferentes partes de manera 
simultanea. En el contexto de EA la entrada serían los tokens de los textos y la salida sería las
estructuras argumentativas anotadas en dependencia del algoritmo usado. Este enfoque mitiga las posibles
deficiencias del enfoque por etapas, al juntar todo el proceso en una sola eliminando la propagación
del error, además de que al tener todos los datos es posible encontrar mayor cantidad de correlaciones 
entre ellos, también no requiere de una ingeniería de atributos tan elaborada [\cite{eger2017neural}].

En la práctica la EA tiene un gran número de aplicaciones [\cite{janier2019argument}]: 
\begin{itemize}
    \item Análisis de opinión: Ayuda no solo a saber si la opinión es favorable o no, sino a saber
    porqué es favorable o no.
    \item Análisis de debate: Ayuda a detectar estrategias argumentativas
    \item Detección de incoherencias en un conjunto de argumentos y justificaciones
\end{itemize}

% Actualidad, novedad e importancia

Desde el punto de vista del proyecto CORESPUC, este trabajo añade la capacidad de anotar
sus textos con las estructuras argumentativas pertinentes ampliando cantidad de
información anotada en este. Desde el punto de vista de Extracción de Argumentos, supone una
adición al estudio de este campo en el lenguaje español, del cual se encontraron pocas
investigaciones realizadas [\cite{esteve2020mineria}]. Con respecto a Cuba en específico
resulta un trabajo completamente nuevo según las investigaciones hechas.

% Pregunta científica o hipótesis

Para la solución del problema se necesita extraer las componentes argumentativas con sus relaciones
y clasificaciones. Para esto se necesita encontrar qué modelo es factible usar para la extracción de
argumentos en textos en español, especialmente en la prensa. En este aspecto existen investigaciones
en las cuales usan modelos basados en Transformer y Attention que han impuesto nuevos estados de arte
[\cite{mayer2020transformer}, \cite{galassi2018argumentative}].

% Este trabajo necesita encontrar \emph{qué modelos se pueden usar en el español para la extracción de 
% argumentos en textos, especialmente en la prensa} (TODO Posible Pregunta Científica?). Recientemente se ha introducido modelos basados 
% en Transformer y Attention (CITE \cite{mayer2020transformer}, \cite{galassi2018argumentative})
% que han llegado a alcanzar resultados iguales o superiores al estado del arte de su momento (TODO Posible Hipótesis?). 
% Por otro lado dado que los corpus están en inglés se desea saber si \emph{es factible usar métodos
% para poder usar el conocimiento aprendido de los algoritmos entrenados en inglés en el español} 
% (TODO Posible Pregunta Científica).

% Objetivos 

El objetivo principal de este trabajo fue el diseño e implementación de un algoritmo para 
el estudio de la argumentación en el periódico digital Granma. Para esto primero
fue necesaria la construcción de un corpus sobre el periódico. Se recolectaron los documentos
del periódico mediante técnicas de scrapping para conformar un corpus inicial no anotado. Este
fue anotado con las estructuras argumentativas por el modelo previamente entrenado usando técnicas 
de proyección entre lenguajes [\cite{eger2018cross}]. Además de las anotaciones anteriores, se agregarán 
otras, como partes de la oración y entidades nombradas para finalmente añadir el corpus a CORESPUC.

% Estructura del trabajo

El trabajo está conformado por \dots (TODO Poner la estructura del trabajo)


% Esqueleto
% \begin{itemize}
%     \item Hablar sobre el conocimiento y el pensamiento del ser humano como ser racional.
%     No existe una verdad única, si no diferente tipos de verdades para diferentes grupos de personas.
%     Cada grupo de personas presentan argumentos por los cuales creen esas verdades y no creen otras.
%     \item Introducir el tema de la argumentación en el NLP, su usos actuales e importancia.
%     \item Introducción de la problemática (Formar un corpus de Granma con estructuras argumentativas),
%     el porqué se quiere hacer esto (Justificación del proyecto CORESPUC, crear un estudio en español del tema)
%     \item Hablar sobre la importancia teórica y práctica del trabajo. No existen estudios en español,
%     los corpus en español son escasos. Ayuda a resolver la justificación de CORESPUC
%     \item Planteo de los objetivos y las preguntas científicas (Crear corpus argumentativo en Español y un framework para la extracción de argumentos en periódicos)
%     \item Estructura del trabajo
% \end{itemize}


% Aspectos que debe tratar la introducción (Se deben de decir implícitamente en los párrafos):

% \begin{itemize}

%     \item Contexto histórico-social donde se desarrolla
%     \item Antecedentes del problema, justificación y motivación. Cómo se ha estudiado primero a mano y luego computacionalmente el problema en la prensa. Motivacion, el proyecto esta integrado en un proyecto nacional  reconocido CORESPUC, lo cual tiene una justificación también.
%     \item Breve presentación de la problemática. (No es el estado del arte aunque se puede hablar un poco de él) Elementos involucrados en el punto de vista cientifico, lleva corpus.
%     \item Actualidad, novedad e importancia teórica y práctica. Revisar literatura (Actualidad, en español no tiene mucho estudio), Se propone un modelo computacional para estudiar ese asutnto que se han propuesto poco, para Cuba no hay ninguno y poco de investigación. 
%     \item Diseño teórico.

%     \begin{itemize}

%         \item Problema: 
%         \item Objeto de Investigación: Procesamiento de Lenguaje Natural
%         \item Campo de acción: Linguistica computacional
%         \item Hipótesis o preguntas científicas
%         \item Objetivos generales y específicos

%         \begin{itemize}
%             \item General
%             \begin{itemize}
%                 \item Diseño e implementación de un algoritmo para el estudio de la argumentación en el periódico digital Granma
%             \end{itemize}
%             \item Específicos
%             \begin{itemize}
%                 \item Construcción del corpus de los periódicos: Crawler, Anotación (scpaCy).
%                 \item Arreglar etiquetas en corpus activo con archivos VRT.
%                 \item Clasificar las secciones en de opinión o no.
%                 \item Verificar la aparición o no de noticias entre la versión PDF y la versión online del periódico.
%                 \item Implementacion de la interfaz gráfica para consultar los resultados.
%                 \item Lograr interoperailidad de la plataforma CQPweb.
%             \end{itemize}
%         \end{itemize}

%     \end{itemize}

%     \item Estructura del trabajo

% \end{itemize}
\chapter{Argumentación}\label{chapter:argumentation}

En el captítulo se aborda la argumentación, definiciones y marcos teóricos existentes para el estudio de esta.
Luego se introduce la Extracción de Argumentos como campo de la lingüísticas computacional encargado del estudio 
y procesamiento de las estructuras argumentativas de textos. También se definen y explican los componentes y las tareas 
asociadas al problema de realizar la extracción de argumentos.

\section{Argumentación}

La argumentación es un tema tratado desde la antigüedad, Aristóteles lo defendía como la 
habilidad de, dada una pregunta, considerar los elementos útiles para persuadir a alguien, algo
similar a la retórica. De una perspectiva más contemporánea surgen las ideas de 
\textcite{perelman1969rhetoric}
enfocadas en un análisis de la retórica en donde se estipula que la teoría de la argumentación
responde a provocar o aumentar la adhesión de las personas a las tesis presentadas, por medio de 
técnicas discursivas. En 
\textcite{toulmin_2003}
se considera como argumento todo aquello que ofrece, 
o todo lo que es utilizado, para justificar o refutar una proposición. En este último, se toma 
una perspectiva más racional y deductiva de la argumentación, dando como resultado lo que se 
conoce como el Método de Toulmin. 

\subsection{Método de Toulmin}

Este método divide los argumentos en seis partes: afirmación 
(\emph{claim}), fundamento (\emph{grounds}), justificación (\emph{warrant}), calificador 
(\emph{qualifier}), refutación (\emph{rebuttal}) y respaldo (\emph{backing}).
Mediante las afirmaciones se conoce el argumento principal que el autor quiere probar a la audiencia,
estas son respaldadas con fundamentos, siendo estos las evidencias y hechos en que se apoya el autor.
Las justificaciones pueden estar explícitas o implícitas y son suposiciones que vinculan los
fundamentos con las afirmaciones, estas a su vez pueden ser respaldadas por conocimiento.
El esquema introduce la posibilidad de otra situación válida a la establecida en las afirmaciones
mediante la refutación. Los calificadores son usados para dar más información de la calidad o seguridad
de las afirmaciones dadas. Un ejemplo\footnote{Extraído de  
\textcite{toulminArgument}.
} de este esquema es:

\begin{adjustwidth}{25pt}{25pt}
    [\emph{Se escucharon ladridos y aullidos en la distancia}]$_{\mathrm{fundamento}}$, 
    [\emph{probablemente}]$_{\mathrm{calificador}}$ 
    [\emph{haya perros en las cercanías}]$_{\mathrm{afirmación}}$.
\end{adjustwidth}

En este ejemplo, además de las partes explícitas, se encuentran partes implícitas como la justificación 
(\emph{los perros son animales que ladran y aúllan}), el respaldo (\emph{se sabe que existen perros en la zona}) y 
la refutación (\emph{puede ser que hayan lobos o coyotes cerca}).

Este método crea una definición compacta que ayuda a los investigadores a enfocar su búsqueda 
en las diferentes categorías definidas. Además, engloba de manera comprensible un tema tan complejo 
como la argumentación al tomar en cuenta gran parte de los elementos presentes en el razonamiento
realizado para llegar a conclusiones, incorporando incluso elementos probabilísticos en el proceso. 

\subsection{Rasgos lingüísticos}

Los rasgos lingüísticos son aquellas características que se encuentran presentes en los textos 
que hacen que estos se clasifiquen argumentativos [\cite{venegas2005hacia}]. Con 
la identificación de estos se hace la tarea de extracción más sencilla y con un marco teórico 
que respalde las decisiones tomadas. Ejemplos de rasgos presentes en textos argumentativos:

\begin{enumerate}
    \item Marcas de orden que introducen párrafos: \emph{primero}, \emph{segundo}, \emph{por un lado}, 
    \emph{por otra parte}, \emph{finalmente}.
    \item Comillas y citas: citar palabras que refuercen la intervención recurriendo a autoridades
    o personajes.
    \item Nexos que expresan causa o consecuencias: \emph{ya que}, \emph{porque}, \emph{pues}, 
    \emph{con motivo de}, \emph{gracias a}, \emph{considerando que}, \emph{por lo tanto}, \emph{de manera que}.
\end{enumerate}

Estos rasgos además de dar indicación de la existencia de argumentos dan pie para conocer las relaciones
entre estos y los tipos de argumentos. Por ejemplo, \emph{por lo tanto}, implica que lo que viene 
a continuación es una conclusión apoyada en lo dicho anteriormente en el texto. Algo parecido
sucede con \emph{ya que}, en este caso implica que lo siguiente es un argumento que se encuentra 
relacionado con lo mencionado antes.

En
\textcite{venegas2005hacia}
se determinan 16 categorías y 51 rasgos lingüísticos, dando una idea 
de la gran variedad de marcadores presentes en la argumentación.

\section{Extracción de Argumentos}

El PLN es un subcampo de la Inteligencia Artificial que tiene como objetivo la comprensión 
del lenguaje humano por las computadoras. 
Mediante el uso de sus algoritmos es posible el procesamiento masivo de texto para la extracción de información 
relevante de este. Entre las tareas pertenecientes a dicho campo se encuentran la Traducción Automática, 
la Generación de Lenguaje Natural y la Extracción de Argumentos (EA). La EA constituye la identificación y extracción 
automática de las estructuras de inferencia y 
razonamiento expresadas como argumentos presentes en el lenguaje natural [\cite{lawrence2020argument}].
En la actualidad, tareas del PLN como el análisis de sentimientos permiten 
extraer cuáles son las opiniones o sentimientos presentes, sin embargo, este análisis presenta una falta 
de información, ya que no justifica el porqué de las opiniones. La EA permite dar respuesta a este problema presentando
los argumentos y cómo sus relaciones justifican las posiciones del hablante. Dicho problema está constituido por diferentes 
estructuras y se compone de distintas tareas necesarias para su solución.

\subsection{Estructuras Argumentativas}

Las estructuras argumentativas son las partes de la argumentación de los textos y sus relaciones.
Estas se componen de dos elementos principales: las Unidades de Discurso Argumentativas (UDA) y los enlaces
existentes entre estas. Las UDAs corresponden a la unidad mínima de argumentación, definida 
como un segmento de texto que juega un solo rol para el argumento analizado, y es 
delimitado por segmentos vecinos que tienen roles diferentes o ningún rol [\cite{stede2018argumentation}].
Las UDAs se relacionan entre sí conformando el proceso de inferencia y razonamiento del argumento.
Tanto los enlaces como las UDAs son clasificados en dependencia de su rol en la argumentación, estas clasificaciones 
en las UDAs parten de los conceptos de afirmación, declaración controversial y parte central del argumento, y premisa,
razones que la justifican o refutan, y en las relaciones de ataque y apoyo. 

\subsection{Tareas de extracción de argumentos}

Dada la definición de estructuras argumentativas y que el objetivo de la EA es extraerlas,
se conciben las siguientes tareas principales:

\subsubsection{Extracción de UDAs}

Consiste en la separación de los segmentos de texto que formarán parte de la estructura.
En este proceso el texto es segmentado y se obtiene un conjunto de UDAs. En el siguiente 
ejemplo\footnote{Traducido del corpus de 
\textcite{stab2017parsing}
.} se representa 
la extracción de UDAs marcadas en \emph{cursiva} de un texto dado:

\begin{adjustwidth}{25pt}{25pt}
    En primer lugar, [\emph{el correo electrónico puede contar como uno de los resultados
    más beneficiosos de la tecnología moderna}]. [\emph{Años atrás, las personas pagaban gran cantidad de dinero para 
    enviar sus cartas y sus pagos estaban sujetos al peso de sus cartas o paquetes y muchos accidentes podrían 
    causar problemas que causarían que el correo no fuera enviado}].
\end{adjustwidth}

\subsubsection{Clasificación de UDAs}

La clasificación de UDAs consiste en asignarle la categoría que toma la UDA en la argumentación. En general, 
las clasificaciones parten de dos clases bases, las afirmaciones y premisas, aunque estas pueden ser tantas
como sea necesario por el problema específico a tratar. Siguiendo con el ejemplo, se observa la clasificación
en afirmación y premisa asignada a los segmentos extraídos en el paso anterior:

\begin{adjustwidth}{25pt}{25pt}
    En primer lugar, [\emph{el correo electrónico puede contar como uno de los resultados
    más beneficiosos de la tecnología moderna}]$_{\mathrm{Afirmación}}$. [\emph{Años atrás, las personas pagaban gran cantidad de dinero para 
    enviar sus cartas y sus pagos estaban sujetos al peso de sus cartas o paquetes y muchos accidentes podrían 
    causar problemas que causarían que el correo no fuera enviado}]$_{\mathrm{Premisa}}$.
\end{adjustwidth}

\subsubsection{Extracción de relaciones entre las UDAs}

La extracción de relaciones constituye el paso donde se determina si están relacionadas las UDAs o no. 
La disposición de estas relaciones forma el proceso de razonamiento en que se basa el autor para validar 
su posición. En el ejemplo se representa la existencia de relación mediante su distancia argumentativa con 
la UDA con la que se relaciona. La distancia argumentativa es la cantidad de UDAs del texto que separan la 
UDA fuente del objetivo [\cite{galassi2021deep}], en caso de ser negativa (positiva) el objetivo se encuentra 
antes (después) que la fuente.

\begin{adjustwidth}{25pt}{25pt}
    En primer lugar, [\emph{el correo electrónico puede contar como uno de los resultados
    más beneficiosos de la tecnología moderna}]$_{\mathrm{Afirmación}}$. [\emph{Años atrás, las personas pagaban gran cantidad de dinero para 
    enviar sus cartas y sus pagos estaban sujetos al peso de sus cartas o paquetes y muchos accidentes podrían 
    causar problemas que causarían que el correo no fuera enviado}]$_{\mathrm{Premisa, -1}}$.
\end{adjustwidth}

\subsubsection{Clasificación de relaciones entre las UDAs}

La clasificación de las relaciones consiste en clasificar las relaciones extraídas en las categorías pertinentes.
Los tipos de relaciones, nacen de dos clases bases por lo general, las relaciones de apoyo y de ataque.
Las de apoyo son aquellas en las que la UDA fuente afirme la UDA objetivo, las de ataque son 
las que la UDA fuente apoya la negación de la UDA objetivo.

\begin{adjustwidth}{25pt}{25pt}
    En primer lugar, [\emph{el correo electrónico puede contar como uno de los resultados
    más beneficiosos de la tecnología moderna}]$_{\mathrm{Afirmación}}$. [\emph{Años atrás, las personas pagaban gran cantidad de dinero para 
    enviar sus cartas y sus pagos estaban sujetos al peso de sus cartas o paquetes y muchos accidentes podrían 
    causar problemas que causarían que el correo no fuera enviado}]$_{\mathrm{Premisa, -1, apoyo}}$.
\end{adjustwidth}

Partiendo de esto, se puede observar que las estructuras argumentativas de un texto constituyen un grafo dirigido 
en donde sus nodos representan las UDAs y están anotados con su tipo, y sus vértices representan las 
relaciones entre las UDAs. Dichos vértices se anotan con el tipo de relación existente entra ambas 
(Figura \ref{fig:arg_struct}).

\begin{figure}[h!]
	\begin{center}
		\begin{center}
            \includesvg[scale=.7]{Graphics/Estructuras_argumentativas.svg}
        \end{center}
        % TODO Anotación {y sus relaciones} reduntante, la definición de estructuras argumentativa ya las engloba 
	    \caption{Estructuras Argumentativas.}
        \label{fig:arg_struct}
	\end{center}
\end{figure}

\chapter{Marco teórico}\label{chapter:background}

En este capítulo se introducen conceptos de Aprendizaje Automático (AA). Temas
como la representación de datos, arquitecturas y evaluación de modelos son tratados en sus secciones.
Luego, se hace un recorrido por las diferentes investigaciones recientes realizadas en la EA, viendo
los enfoques desarrollados y la evolución de estos a partir del desarrollo e introducción de nuevos métodos en el
campo. Por último, se introduce la proyección de corpus, técnica necesaria para la creación de conjuntos 
de datos para el entrenamiento de modelos de AA.

\section{Aprendizaje Automático}

Existen varios problemas para los cuales es difícil escribir un programa de forma tradicional que pueda
resolverlo, por ejemplo, decir si en una imagen existe un gato o un perro, o transcribir una grabación. En el caso
de que se escribiera este programa probablemente sería frágil y poco escalable. El AA constituye el 
estudio de técnicas que pueden aprender de la experiencia [\cite{d2l}] y mediante su uso 
se puede automatizar el proceso de encontrar soluciones a dichos problemas, haciendo que sus resultados sean 
robustos y escalables. 

En AA existen tres tipos de aprendizaje:
\begin{itemize}
	\item Aprendizaje no supervisado: trabaja con datos no anotados y los algoritmos tratan de 
	extraer información de estos.
	\item Aprendizaje reforzado: trabaja en la creación de agentes que interactúen con 
	un ambiente obteniendo información de qué tan buenas son las acciones realizadas.
	\item Aprendizaje supervisado: trabaja con datos anotados y los algoritmos tratan de minimizar
	el error cometido en las predicciones.
\end{itemize}
A este último principalmente se refiere esta sección.

Los componentes básicos de un problema de aprendizaje supervisado se pueden resumir en los datos de los que 
aprender, el modelo para transformar los datos, una función de pérdida que cuantifica qué tan malo o bueno es el 
modelo y un algoritmo que ajuste los parámetros del modelo para minimizar la pérdida.

Matemáticamente, un modelo constituye una función $\hat{y} = f_{\theta}(x)$ cuyo argumento son las 
representaciones de las entradas originales $x$ y devuelve una salida $\hat{y}$. Esta función $f$ se encuentra
parametrizada por el vector $\theta$ el cual constituye los parámetros en los que el modelo se basa para realizar sus
cálculos. La función de pérdida se puede definir como $e(y, \hat{y})$, donde $y$ es el resultado esperado para $x$. 
El objetivo final del algoritmo de ajuste es encontrar $\theta$ tal que:

\begin{equation}
	\arg \min_{\theta} e(y, f_{\theta}(x)).
\end{equation}\label{eq:arg_min_theta}

Los parámetros $\theta$ eventualmente se van ajustando con algún algoritmo de optimización para que 
la función de error alcance un valor óptimo, aunque un óptimo global en la práctica es difícil de conseguir
debido a la complejidad del espacio de búsqueda.

Una manera de observar la complejidad de estos modelos sería la cantidad de capas de procesamiento que lo integran.
Los primeros modelos empezaron utilizando una sola capa\footnote{A estos modelos se les conoce 
como \emph{shallow} o poco profundos.} para realizar 
las tareas, más tarde estos modelos fueron ganando en complejidad al superponerle otras, dando paso al 
aprendizaje profundo (\emph{Deep Learning}). La superposición se refiere
a la composición de funciones $f_i$ y $f_{i+1}$ de tal forma que la imagen de $f_i$ sea compatible con el dominio de 
$f_{i+1}$, entonces se obtiene $f_{i+2}(x) = f_{i+1}(f_i(x))$, al aplicar este esquema se pueden agregar múltiples
capas y profundizar $f$ tanto como se desee.


\subsection{Representación de datos}

La representación de los datos constituye una parte importante al modelar un problema. Esta 
es la encargada de presentar datos abstractos, como imágenes, párrafos o sonidos, en formas tratables
por los algoritmos. Generalmente, se buscan configuraciones que recojan la mayor cantidad de información 
de la entrada relevante al problema.

En el PLN los datos suele estar representados con distintos niveles de granularidad.
De menor granularidad a mayor se pueden ir mencionando los documentos, párrafos, oraciones, palabras y
caracteres. Estos elementos es posible representarlos mediante vectores que codifiquen propiedades
objetivas del problema a tratar como características morfológicas o semánticas de estos. Por ejemplo, 
los textos tienen una popular manera de representarse mediante vectores de TF-IDF [\cite{manning2008introduction}],
una desventaja de esta representación es que no toma en cuenta el orden de las palabras, por lo que 
la representación de \emph{no me gusta,} y \emph{no, me gusta} serían iguales aunque semánticamente 
sean opuestas. Para agregarle la información de orden a la representación se introducen los llamados 
$n-$gramas, los cuales toman en cuenta una ventana de tamaño $n$ para conformar conjuntamente una representación.
Este enfoque conlleva a la limitante de que solo un contexto finito está disponible para hacer las inferencias,
para esto finalmente se crean representaciones individuales para cada palabra, las cuales codifican 
información y la secuencia completa es introducida al modelo para realizar la inferencia.
La manera más simple de representar una palabra es mediante la representación \emph{one-hot}, 
en esta, a cada palabra se le asigna un índice y el vector resultante posee la dimensión del 
vocabulario y es rellenado con ceros en todos sus elementos excepto en el índice de la palabra, donde se le asigna 1.
Esta representación asume que las palabras son independientes entre sí y 
computacionalmente ocupan un espacio considerable. Existen otras representaciones 
que brindan información al algoritmo sobre la morfología y la semántica de la palabra que representa.

Las características morfológicas son aquellas que describen cómo está formado el elemento a analizar.
Estas pueden ser extraídas con relativa facilidad, entre tales características se encuentran: tamaño, 
cantidad, posición de palabras o párrafos, presencia de sufijos, prefijos, acentos u otros marcadores
en el texto. Las características semánticas presentan una mayor dificultad a la hora de ser extraídas.
Para esto se utilizan diferentes modelos que codifican esta información en vectores continuos conocidos como 
\emph{embeddings}, entre los modelos existentes se encuentran
word2vec [\cite{mikolov2013efficient}], 
\emph{Global Vectors} (GloVe) [\cite{pennington2014glove}], 
\emph{Bidirectional Encoder Representations from Transformers} (BERT) [\cite{devlin2018bert}],
entre otros.

\subsection{Modelación de problemas}

Los problemas en la vida real se presentan en un principio como una descripción a un nivel de abstracción 
más alto que el requerido por los algoritmos existentes, se hablan de entidades abstractas como imágenes,
sonidos, textos o usuarios, además de la información que se desea extraer o las operaciones que se
desean aplicar sobre estas entidades. A lo largo del desarrollo del AA han aparecido diferentes tipos de problemas 
que aparecen frecuentemente en la práctica y sus respectivas maneras de solucionarlos. Uno de estos tipos de problemas
es el de secuencia a secuencia (\emph{seq2seq}), cuyo objetivo es la conversión de una secuencia en un dominio de entrada a otra 
secuencia en un dominio de salida. Con este tipo de problemas se modelan tareas como traducción automática, en 
donde el dominio de entrada es texto en un lenguaje de partida y el dominio de salida es texto en un lenguaje de llegada.
Otro problema que se resuelve con \emph{seq2seq} es la segmentación de texto, cuya entrada es una lista de \emph{tokens}
y la salida son etiquetas que indican cuándo empieza y termina un segmento de texto. Las etiquetas utilizadas para 
la segmentación, generalmente, son:

\begin{itemize}
	\item B (\emph{begin}): representa el inicio del segmento.
	\item I (\emph{inside}): representa la continuación del segmento previamente iniciado.
	\item O (\emph{outside}): representa la no pertenencia a un segmento.
\end{itemize}

Este esquema presenta una versión más elaborada en la que se agregan dos clases que representan el final 
de un segmento E (\emph{end}) y un segmento de un solo elemento S (\emph{single}). La tarea anterior es 
utilizada como base para otra un poco más compleja, en la que, además de segmentar el texto, se desea 
clasificar sus segmentos. Para este problema se añaden meta etiquetas correspondientes a los tipos a identificar $C$ 
% TODO Anotación {no dices quien es O} se dice arriba: representa la no pertenencia a un segmento.
obteniendo un conjunto final $R = \{ B, I, E, S \} \times C \cup \{ O \}$. En tareas como la extracción de entidades
nombradas este método es empleado, donde el conjunto $C$ contiene las posibles categorías de entidades a identificar.

\subsection{Arquitecturas}

En aprendizaje profundo existen una gran cantidad de arquitecturas que se pueden utilizar para formar el modelo 
final. Estas deben de ser seleccionadas en dependencia de los datos y el problema a tratar, ya que sus diseños 
emulan diferentes operaciones sobre datos que pueden ser más beneficiosas en situaciones específicas.

\subsubsection{Capas densas}

El perceptrón consiste es una transformación lineal del vector de datos $\textbf{x}$ con un sesgo $b$ y 
luego aplicar una transformación no lineal $g$, conocida como función de activación, 
para la obtención del resultado final:

\begin{equation}
	f(\textbf{x}) = g(\textbf{w}\textbf{x} + b).
\end{equation}

El perceptrón constituye la unidad básica de las capas densas, ya que estas consisten en la aplicación
de este modelo varias veces sobre la misma entrada $\textbf{x}$ produciendo vectores de la dimensión $k$ 
deseada como salida final. Los parámetros se codifican en la matriz $\textbf{W}$ y sus sesgo en el 
vector $\textbf{b}$:

\begin{equation}
	f(\textbf{x}) = g(\textbf{Wx} + \textbf{b}).
\end{equation}

Para las funciones de activación existen varias elecciones. Una de estas es la función sigmoidal. 
Esta devuelve un valor entre 0 y 1, es usada en tareas de regresión logística. 
Se puede interpretar como el nivel de activación de la neurona:

\begin{equation}
	sigm(x) = \frac{1}{1-e^{-x}}.
\end{equation}

\emph{Rectified Linear Unit} (ReLU) se define como la parte positiva del argumento. Una ventaja que trae esta 
función de activación es su rápido cálculo de su derivada y que previene en parte de los problemas 
de desaparición de gradiente de la sigmoidal:

\begin{equation}
	relu(x) = \max(0, x).
\end{equation}

La función de activación \emph{softmax} es diferente a las anteriores en el sentido que necesita
el vector salida de la capa anterior para ser computada. Esta función convierte las $q$ salidas
en una distribución de probabilidad, algo necesario en tareas de clasificación:

\begin{equation}
	softmax_k(\textbf{x}) = \frac{e^{x_k}}{\sum\limits_{i=1}^{q} e^{x_i}}.
\end{equation}

\subsubsection{Redes Neuronales Convolucionales}

Las redes neuronales convolucionales (CNN, en inglés) 
son un tipo de redes usadas
principalmente para tratar datos donde su estructura espacial es relevante, por ejemplo,
en datos bidimensionales como imágenes y en unidimensionales como sonido y texto.
Estas redes aplican de una función de kernel sobre los datos, $f * g$ donde $f$ son los
datos, $g$ es el kernel o filtro y $*$ es el operador de convolución [\cite{d2l}]. La función $g$ se puede 
aprender en el proceso o también puede ser una función de agrupación predefinida. 
Un ejemplo bidimensional de una función de kernel se muestra en la Figura \ref{fig:conv_kernel}.

\begin{figure}[h!]
	\begin{center}
		\begin{center}
			\includegraphics[scale=.4]{Graphics/kernel_convolution.png}
        \end{center}
		% TODO Cite problem, citar como texto
		% [\autocite{d2l}] 
		% [\parencite{d2l}] 
		% [\citet{d2l}] 
		% [\citep{d2l}] 
	    \caption{Convolución con kernel (Tomado de \textcite{d2l} pág. 241).}\label{fig:conv_kernel}
	\end{center}
\end{figure}

En este se observa cómo una nueva representación es computada al operar el kernel por la matriz de datos. Este
corrimiento se puede realizar de diferentes formas, por ejemplo, se puede mover de dos en dos en vez de uno en uno, este
parámetro se le conoce como tamaño de paso (\emph{stride}). Es posible además preservar las dimensiones
iniciales de los datos al aplicarle un aumento de los datos en los bordes de tal forma que el resultado sea de la misma
dimensión, este aumento se realiza, generalmente, rellenando convenientemente los espacios con ceros, 
a esto se le llama \emph{padding} (Figura \ref{fig:conv_kernel_padding}).

\begin{figure}[h!]
	\begin{center}
		\begin{center}
			\includegraphics[scale=.4]{Graphics/kernel_convolution_padding.png}
        \end{center}
		% TODO Cite problem, citar como texto
		% [\autocite{d2l}] 
		% [\parencite{d2l}] 
		% [\citet{d2l}] 
		% [\citep{d2l}] 
		\caption{Convolución con kernel con \emph{padding} (Tomado de \textcite{d2l} pág. 241).}\label{fig:conv_kernel_padding}
	\end{center}
\end{figure}

Existen varias funciones de agrupación usadas. Entre estas se encuentran las de agrupación máxima y de 
agrupación media. Como sus nombres indican, la de agrupación máxima devuelve el valor máximo de los encontrados
en la ventana del kernel, la de media calcula el promedio de estos valores. Estas capas tienen la capacidad de obtener
información resumida sobre los datos.

\subsubsection{Redes Residuales}

Al crear modelos de aprendizaje profundo se tienen un conjunto de parámetros $\theta$. Las posibles combinaciones 
de estos forman un espacio de funciones $F$ al cual pertenecen todas las posibles instancias del modelo.
Agregar nuevas capas aumenta la complejidad de este, pero no hay garantía de que el viejo espacio 
de funciones $F$ sea subconjunto del nuevo espacio $F'$, lo que implica que el nuevo modelo no es necesariamente
estrictamente superior al antiguo. Este problema es la razón para la aparición de las Redes Residuales. 
Una red residual está formada por 
uno o varios bloques residuales, en los que a la salida de cada bloque residual le es sumada la entrada de 
este mediante una conexión residual.
El objetivo de realizar tal operación es que es posible hacer la contribución del bloque 0 obteniendo así
un modelo equivalente a uno sin el bloque, garantizando la condición de subconjunto $F \subset F'$, además,
dicho bloque no pierde poder expresivo, dado que en caso de que su aporte al resultado final sea considerable, 
se tendría que aprender solamente la función $f(x) - x$ donde $x$ es la entrada del bloque y $f$ es la función 
aprendida por el bloque sin la conexión residual, para mitigar el efecto de la conexión residual como se muestra
en la Figura \ref{fig:res_block}.

\begin{figure}[h!]
	\begin{center}
		\begin{center}
			\includegraphics[scale=.4]{Graphics/resnet.png}
        \end{center}
		% TODO Cite problem, citar como texto
		% [\autocite{d2l}] 
		% [\parencite{d2l}] 
		% [\citet{d2l}] 
		% [\citep{d2l}] 
	    \caption{Bloque residual (Tomado de \textcite{d2l} pág. 289).}\label{fig:res_block}
	\end{center}
\end{figure}

\subsubsection{Redes Neuronales Recurrentes}

Las redes neuronales recurrentes (RNN, en inglés) son
un tipo especial de arquitectura especializada en el trabajo con datos secuenciales. Este tipo de arquitectura
presenta variables en las que se almacenan información pasada, que es usada para el computo de la salida. El 
problema se puede modelar probabilísticamente mediante la estimación de $P(x_t | x_{t-1}, \dots, x_{1})$,
donde existen dos variantes principales. En una variante se fija un tamaño de ventana $\alpha$ en el tiempo, 
dando como resultado $P(x_t | x_{t-1}, \dots, x_{t-\alpha})$, a este tipo de modelos se les conoce como autorregresivos. 
Otra estrategia consiste en guardar un contexto de observaciones pasadas $h_t$ y con este realizar la estimación 
$P(x_t | h_t)$, el contexto se actualiza en cada paso mediante una función $h_t = g(h_{t-1}, x_{t-1})$, a estos 
se les nombra modelos autorregresivos latentes, debido a la existencia de variables ocultas $h_t$. 

En la práctica estos modelos presentan problemas de gradientes, ya que estas pueden volverse extremadamente grandes o 
desaparecer.
Para esto se han creado arquitecturas que disminuyen estos problemas. Una de estas arquitecturas es las de memorias
de corto largo plazo (LSTM, en inglés) [\cite{hochreiter1997long}].
Este modelo guarda un contexto del procesamiento y está constituido por varias compuertas que regulan las 
actualizaciones de los estados internos. LSTM (Figura \ref{fig:rnn_lstm}) posee dos variables de estado, la memoria 
$C$ y el estado oculto $H$. Entre sus compuertas se encuentran la compuerta de olvido, esta regula cuánto de la 
memoria permanece en el próximo paso, la compuerta de entrada ajusta la cantidad de información nueva que entrará, 
la compuerta de salida maneja el cálculo del próximo estado oculto.

\begin{figure}[h!]
	\begin{center}
		\begin{center}
			\includegraphics[scale=.4]{Graphics/rnn_lstm.png}
        \end{center}
		% TODO Cite problem, citar como texto
		% [\autocite{d2l}] 
		% [\parencite{d2l}] 
		% [\citet{d2l}] 
		% [\citep{d2l}] 
	    \caption{LSTM (Tomado de \textcite{d2l} pág. 357).}\label{fig:rnn_lstm}
	\end{center}
\end{figure}

El método de aprendizaje de las RNN solamente observa los elementos anteriores de la secuencia, aunque existen
tareas en las que, observando los elementos posteriores, se brinda más contexto e información a la tarea, sin que interfiera
en el proceso de inferencia. El modelo bidireccional presenta una alternativa para tratar con este tipo de problemas, 
este modelo consiste en, además de hacer el recorrido de inicio a final de la secuencia, realizar otro recorrido en orden 
inverso (Figura \ref{fig:rnn_bidirectional}), estos recorridos van generando dos estados ocultos $\overrightarrow{H}_{i}$ y $\overleftarrow{H}_{i}$
que luego son mezclados para obtener el contexto final $H_i$, los tipos de mezclas comunes son la concatenación de los 
estados o la multiplicación elemento a elemento de estos.

\begin{figure}[h!]
	\begin{center}
		\begin{center}
			\includegraphics[scale=.3]{Graphics/rnn_bidirectional.png}
        \end{center}
		% TODO Cite problem, citar como texto
		% [\autocite{d2l}] 
		% [\parencite{d2l}] 
		% [\citet{d2l}] 
		% [\citep{d2l}] 
	    \caption{Red neuronal bidireccional (Tomado de \textcite{d2l} pág. 367).}\label{fig:rnn_bidirectional}
	\end{center}
\end{figure}


\subsubsection{Atención}

% Definicion, qué hace, uso en secuencias
La atención es una técnica en la cual se hace una selección ponderada de atributos en un contexto específico. 
Este mecanismo presenta dos partes, una consulta $q$ y una colección de pares llave-valor, $(k_i, v_i)$. La 
consulta representa el contexto en donde se quiere aplicar la atención y las llaves $k_i$ son elementos que 
relacionan la consulta a los valores $v_i$. El proceso de calcular el resultado consiste en, primero, calcular 
el vector compatibilidad $e$ entre las llaves y la consulta mediante la función $f$, este vector es luego 
modificado por una función $g$ que distribuye los valores obteniendo el vector atención $a$. Finalmente 
este vector es utilizado para calcular el resultado final al aplicarle la función $o = z(a, V)$:

\begin{equation}
	e = f(q, K),
\end{equation}
\begin{equation}
	a = g(e),
\end{equation}
\begin{equation}
	o = z(a, V).
\end{equation}

En dependencia de cómo se seleccionen las funciones $f$, $g$ y $z$ se pueden obtener distintos tipos de atención.
Una configuración simple consiste en definir $f$ como el producto punto de la consulta con la llave,
$g$ como \emph{softmax} y $z$ la suma ponderada de $v_i$ con los valores de atención.

\subsubsection{Campo Aleatorio Condicional}

% Definicion de CRF, qué modela, Ventajas de CRF en secuencias, mirar relacion con Hidden Markov Models

El campo aleatorio condicional (CRF, en inglés) es un 
tipo de modelo gráfico probabilístico que trabaja eficientemente con secuencias,
% TODO PREGUNTAR {modelando} se cambia para s y luego de {conjuntamente} se se pone con? 
modelando conjuntamente la probabilidad de las etiquetas de sus elementos dadas sus observaciones [\cite{lafferty2001conditional}].
En trabajos de secuencias, la forma más simple que toma el grafo consiste en una cadena de las variables representando
las etiquetas de las secuencias $Y$, conectadas de la forma $(Y_i, Y_{i+1})$ y las variables observadas $X$, conectadas
a las variables $Y$ [\cite{wallach2004conditional}].

El objetivo de CRF es calcular la secuencia $Y^*$ tal que:

\begin{equation}
	Y^* = \arg \max_Y P(Y | X).
\end{equation}\label{eq:crf}

En esta expresión se observa que devuelve la secuencia más probable, dadas las variables observadas o atributos $X$,
por lo que esta capa es usada al final del proceso para problemas de clasificación de secuencias.

\subsection{Evaluación del modelo y métricas}

Los modelos de AA necesitan maneras de expresar qué tan buenos son 
en las tareas encomendadas. Para esto se crean funciones que evalúan los resultados obtenidos
por dichos modelos, estas funciones se les da el nombre de métricas. Existen diferentes tipos de
métricas para tratar con diferentes tipos de problemas. En aprendizaje supervisado una métrica se
define como una función $m_s(Y, \hat{Y})$, donde $Y$ son las predicciones verdaderas y $\hat{Y}$ son las predicciones
hechas por el modelo. En algoritmos de aprendizaje no supervisado como K-Means y K-NN son usadas funciones $m_{ns}(\hat{Y})$
donde $\hat{Y}$ son las predicciones finales. En comparación con su versión supervisada estas funciones no tiene acceso
a las predicciones verdaderas del problema.

\subsubsection{Clasificación}

En problemas de clasificación (problemas en donde las etiquetas a predecir son discretas) 
son empleadas medidas que toman en cuenta la naturaleza discreta de su conjunto imagen.
Medidas como precisión, recobrado, \emph{accuracy} y F1 son utilizadas en la 
evaluación de los resultados, mientras que como función de error se usa entropía cruzada 
(\emph{cross entropy} en inglés) [\cite{grandini2020metrics}].

La matriz de confusión es una vía de representar los resultados de dos clasificadores. Esta matriz en $M_{ij}$ 
indica la cantidad de elementos que clasificó como clase $i$ el primer clasificador y
como clase $j$ el segundo clasificador. En su uso práctico,
un clasificador son las etiquetas verdaderas mientras que el otro es el clasificador que se está evaluando. 
En problemas de la clasificación binaria, donde se busca saber si existe pertenencia o no de un elemento a una clase,
se pueden observar los siguientes casos:

\begin{itemize}
	\item Verdaderos Positivos (VP): elementos clasificados correctamente que pertenecen a la clase.
	\item Verdaderos Negativos (VN): elementos clasificados correctamente que no pertenecen a la clase.
	\item Falsos Positivos (FP): elementos que no pertenecen a la clase clasificados incorrectamente en que pertenecen.
	\item Falsos Negativos (FN): elementos que pertenecen a la clase clasificados incorrectamente en que no pertenecen.
\end{itemize}

\begin{table}[h!]
	\begin{center}
		\begin{tabular}{|c|c|c|} \hline
		Clases		& Positivo	& Negativo  \\ \hline
		Positivo	& VP  		& FN		\\ \hline
		Negativo	& FP		& VN		\\ \hline
		\end{tabular}
	\caption{Matriz de confusión binaria.}\label{fig:confusion_matrix}
	\end{center}
\end{table}

La precisión es la medida que indica la probabilidad de que la clasificación de una clase sea correcta. Esto 
se puede observar como la proporción de los elementos correctamente clasificados sobre el total de 
elementos clasificados:

\begin{equation}
	prec_i = \frac{VP}{VP + FP}.
\end{equation}

En problemas de clasificación múltiple surge la versión macro de esta medida calculada como la media de todas
las precisiones de las clases existentes:

\begin{equation}
	prec_{macro} = \sum^K_{i=1} \frac{prec_i}{K}.
\end{equation}

El recobrado es la medida que indica la probabilidad de que se clasifique correctamente un elemento de la clase
del total existente. Esto se puede observar como la proporción de los elementos correctamente clasificados sobre el 
total de elementos que pertenecen a la clase:

\begin{equation}
	rec_i = \frac{VP}{VP + FN}.
\end{equation}

En problemas de clasificación múltiple surge la versión macro de esta medida calculada como la media de todos
los recobrados de las clases existentes:

\begin{equation}
	rec_{macro} = \sum^K_{i=1} \frac{rec_i}{K}.
\end{equation}

La medida F1 es la media armónica de la precisión y el recobrado. En esta la contribución de la precisión y el
recobrado al resultado final es el mismo, aunque es posible buscar variaciones de acuerdo a al problema a tratar:

\begin{equation}
	F1_i = 2 \frac{prec_i \cdot rec_i}{prec_i + rec_i}.
\end{equation}

En problemas de clasificación múltiple surge la versión macro de esta medida calculada la propia medida F1, pero
utilizando la precisión y recobrado macro del problema:

\begin{equation}
	F1_{macro} = 2 \frac{prec_{macro} \cdot rec_{macro}}{prec_{macro} + rec_{macro}}.
\end{equation}

La métrica $\alpha$\%F1 [\cite{persing2016end}] es una métrica basada en la idea de F1 orientada para el 
trabajo con secuencias donde $\alpha$ denota el porcentaje de secuencia inferida que debe coincidir con 
la secuencia anotada para ser considerado una coincidencia. Esta versión permite
establecer el rango de flexibilidad si $\alpha=100$ (100\%F1), significa que deben coincidir completamente, 
mientras si $\alpha = 50$ (50\%F1) significa que, si coinciden en una proporción mayor o igual a la mitad, 
se considera como un verdadero positivo. La definición de los valores de verdaderos positivos, negativos 
y falsos negativos es:

\begin{equation}
	VP = |\{ j | \exists i | gl(j) = pl(i) \land i = j \}|,
\end{equation}
\begin{equation}
	FP = |\{ i | pl(i) \neq n \land \not\exists j | gl(j) = pl(i) \land i = j \}|,
\end{equation}
\begin{equation}
	FN = |\{ j | \not\exists i | gl(j) = pl(i) \land i = j \}|,
\end{equation}

donde $i$ y $j$ son las UDAs extraídas, $gl(j)$ es la etiqueta correcta para $j$, $pl(i)$ es 
la etiqueta inferida para $i$, $n$ es la clase no argumentativa, $i = j$ significa que $i$ es 
una coincidencia para $j$.

Las métricas anteriores están acotadas por los valores 0 y 1, donde 1 representa la mejor evaluación y 0 la 
peor.

La entropía cruzada se encarga de evaluar qué tan diferentes son dos funciones de distribución $p$ y $q$, su 
resultado es un número no negativo que, a medida que sean más pequeños los valores, indican mayor similitud. 
En su versión discreta se formula así:

\begin{equation}
	H(p, q) = - \sum_{x \in D} p(x) \log q(x).
\end{equation}

\subsubsection{Cadenas}

Para textos o cadenas existen diferentes métricas que constituyen formas de saber la similitud 
entre dos elementos. Una de esas métricas es la similitud de Jaccard, esa se puede ver como 
la proporción de elementos comunes que presentan dos conjuntos, los elementos sería palabras:

\begin{equation}
	jac(X, Y) = \frac{|X \cap Y|}{|X \cup Y|}.
\end{equation}
Otra medida de similitud es la distancia de Levenshtein, la cual se define como la mínima cantidad 
de cambios de eliminar, cambiar y agregar que se tienen que hacer a dos secuencias para que sean 
iguales. Esta medida se puede usar tanto en palabras, en donde se mediría la cantidad de cambios 
a los caracteres, como en listas de palabras, donde se mediría la cantidad de palabras que 
se tienen que cambiar. 

\subsubsection{Curvas de aprendizaje}

Es necesario además de evaluar el resultado final del modelo, evaluar el proceso de entrenamiento. En esta etapa 
se pueden diagnosticar varias deficiencias en este proceso. Para un correcto entrenamiento se divide el conjunto de 
datos en tres partes:

\begin{itemize}
	\item \textbf{entrenamiento}: utilizada para el entrenamiento del modelo.
	\item \textbf{validación}: utilizada para evaluar el desempeño del modelo durante el entrenamiento.
	\item \textbf{prueba}: utilizada para evaluar el resultado final.
\end{itemize}

Las curvas de aprendizaje constituyen la principal herramienta para evaluar el proceso de aprendizaje.
Estas están formadas por las mediciones de métricas a lo largo del entrenamiento calculadas a partir de 
los conjuntos de validación y entrenamiento. En estas, el eje horizontal representa el número de época del entrenamiento,
iteración sobre el conjunto de entrenamiento, y el eje vertical representa el valor de la métrica a analizar. 
La línea correspondiente al conjunto de entrenamiento cuantifica 
el aprendizaje del modelo o también el error de entrenamiento, y la correspondiente a la de validación cuantifica 
la generalización o el error de generalización. Existen tres comportamientos esenciales a analizar:

\begin{itemize}
	\item Bajo ajuste (\emph{underfitting}).
	\item Sobreajuste (\emph{overfitting}).
	\item Buen ajuste.
\end{itemize}

El bajo ajuste ocurre cuando el modelo no es capaz de aprender del conjunto de datos o cuando este aún puede aprender 
más. Las curvas de aprendizaje en estos casos se caracterizan por ser una línea plana o valores ruidosos con alta pérdida
(Figura \ref{fig:underfit}).

\begin{figure}[h!]
	\begin{center}
		\includegraphics[scale=.2]{Graphics/underfit_missing_training.png}
		\includegraphics[scale=.2]{Graphics/underfit_not_learning.png}
		% TODO Cite problem, citar como texto
		% TODO PREGUNTAR, la página es en romano, luego de esas vienen páginas en números normales,
		% poner el número normal traería confusión
		% [\autocite{brownlee2018better}] 
		% [\parencite{brownlee2018better}] 
		% [\citet{brownlee2018better}] 
		% [\citep{brownlee2018better}] 
		\caption{Curvas de entrenamiento con bajo ajuste por falta de entrenamiento (izquierda) 
		y por modelo que no aprende de los datos (derecha) (Tomado de \textcite{brownlee2018better} pág. XXVII).}\label{fig:underfit}
	\end{center}
\end{figure}

Entre las formas más sencillas de combatir el bajo ajuste de los modelos consiste en complejizarlo, al añadir
capas o aumentar las dimensiones de este aumenta su expresividad y, por lo tanto, su ajuste. Si este método 
no funciona es posible considerar un cambio de arquitectura hacia una que pueda extraer más información de la 
estructura de los datos. 

El sobreajuste es el fenómeno en el que el modelo aprende los datos de entrenamiento extremadamente bien, incluso
el ruido en estos, esto trae consigo que falla en generalizar el problema para nuevas entradas. Las curvas 
características de este fenómeno presentan una divergencia en los errores de entrenamiento y validación a medida
que se entrena el modelo, mientras que la de entrenamiento mejora la de validación tiende a empeorar (Figura \ref{fig:overfit}). 

\begin{figure}[h!]
	\begin{center}
		\begin{center}
			\includegraphics[scale=.3]{Graphics/overfit_raising_val_error.png}
        \end{center}
		% TODO Cite problem, citar como texto
		% TODO PREGUNTAR, la página es en romano, luego de esas vienen páginas en números normales,
		% poner el número normal traería confusión
		% [\autocite{brownlee2018better}] 
		% [\parencite{brownlee2018better}] 
		% [\citet{brownlee2018better}] 
		% [\citep{brownlee2018better}] 
	    \caption{Curvas de entrenamiento con sobreajuste (Tomado de \textcite{brownlee2018better} pág. XXIX).}\label{fig:overfit}
	\end{center}
\end{figure}

Existen varios métodos para combatir el sobreajuste, uno sencillo es simplificar el modelo quitándole capas 
o disminuyendo sus dimensiones. Además de esto, existen regularizaciones que se pueden aplicar para evitar que 
las capas dependan exclusivamente de pocos atributos, entre esta familia los más usados son la regularización
L1 y L2 las cuales se definen como la suma del valor absoluto de los atributos y la suma del cuadrado de sus 
atributos respectivamente. Otra medida para prevenir el sobreajuste es el agrego de capas de abandono 
(\emph{dropout}). Estas capas desactivan neuronas de la arquitectura, obligando a 
estas a ser robustas y depender del comportamiento de la población, en lugar de la actividad de otras unidades 
específicas [\cite{baldi2013dropout}]. La terminación temprana (\emph{early stopping}) del entrenamiento
se utiliza para parar este en el momento en que el error de generalización comienza a subir, impidiendo así que 
se sobreentrene el modelo.

Finalmente, un buen ajuste es el resultado que se alcanza cuando tanto la curva de validación como de entrenamiento
presentan valores pequeños y similares, consecuentes con una correcto aprendizaje y generalización (Figura \ref{fig:good_fit}).

\begin{figure}[h!]
	\begin{center}
		\begin{center}
			\includegraphics[scale=.3]{Graphics/good_fit.png}
        \end{center}
		% TODO Cite problem, citar como texto
		% TODO PREGUNTAR, la página es en romano, luego de esas vienen páginas en números normales,
		% poner el número normal traería confusión
		% [\autocite{brownlee2018better}] 
		% [\parencite{brownlee2018better}] 
		% [\citet{brownlee2018better}] 
		% [\citep{brownlee2018better}] 
	    \caption{Curvas de entrenamiento con buen ajuste (Tomado de \textcite{brownlee2018better} pág. XXX).}\label{fig:good_fit}
	\end{center}
\end{figure}

Otro problema observable a partir del análisis de las curvas de aprendizaje constituye la detección de conjuntos
de datos no representativos. Un conjunto de datos no representativo es uno que puede no 
capturar las características estadísticas relativas a otro conjunto de datos extraído del mismo dominio.
Esto puede pasar que los conjuntos de entrenamiento o de validación. En caso del conjunto de entrenamiento
se puede identificar si la pérdida en el conjunto de entrenamiento conlleva a una ganancia en el conjunto de 
validación y viceversa quedando al final con una separación entre ambos valores. En el caso del conjunto de 
validación se presenta como una curva ruidosa, también se puede dar el caso de que el conjunto  de validación
sea más fácil de predecir que el de entrenamiento, en este caso se observa como la curva de validación permanece
siempre por debajo de la de entrenamiento (Figura \ref{fig:unrepresentative_data}).

\begin{figure}[h!]
	\begin{center}
		\includegraphics[scale=.2]{Graphics/unrepresentative_dev_set.png}
		\includegraphics[scale=.2]{Graphics/unrepresentative_train_set.png}
		% TODO Cite problem, citar como texto
		% TODO PREGUNTAR, la página es en romano, luego de esas vienen páginas en números normales,
		% poner el número normal traería confusión
		% [\autocite{brownlee2018better}] 
		% [\parencite{brownlee2018better}] 
		% [\citet{brownlee2018better}] 
		% [\citep{brownlee2018better}] 
		\caption{Curvas de entrenamiento con datos poco representativos en el conjunto de validación (izquierda) 
		y en el conjunto de entrenamiento (derecha) (Tomado de \textcite{brownlee2018better} pág. XXXII).}\label{fig:unrepresentative_data}
	\end{center}
\end{figure}

Para combatir estos problemas se puede aumentar la cantidad de elementos en los conjuntos de entrenamiento o 
validación en dependencia de donde ocurra.

\subsection{Aumento de datos}

El aumento de datos consiste en acciones para aumentar la diversidad de un conjunto de datos sin recolectar
nuevos datos explícitamente [\cite{feng2021data}]. En datos continuos, como imágenes o valores numéricos el 
aumento de datos puede ser realizado al añadirle perturbaciones a entradas existentes, en caso de las imágenes 
técnicas como el volteado (\emph{flipping}) o recortado (\emph{cropping}) son usadas. Los textos son un tipo 
de datos discreto y, por lo tanto, las técnicas anteriores no pueden ser aplicadas directamente. Para el PLN
se han estudiado diversas técnicas de aumento de datos, una de estas consiste en el cambio del árbol de 
dependencia de la oración mediante operaciones de intercambio y borrado de nodos [\cite{csahin2019data}], también se han utilizado 
el intercambio de palabras por sinónimos [\cite{dai2020analysis}] y la traducción de textos hacia un lenguaje y luego 
de vuelta al lenguaje origen (\emph{backtranslation}) [\cite{sennrich2015improving}]. 

\subsection{Aprendizaje Conjunto}

El aprendizaje conjunto (\emph{ensemble learning}) son técnicas encaminadas al aprovechamiento
de soluciones encontradas por diferentes modelos, combinándolas y mejorándolas para encontrar una mejor solución 
al problema. Estos métodos son efectivos en la reducción de la varianza y el sesgo de los modelos, obteniendo así
mejores resultados [\cite{dietterich2002ensemble}]. Una forma de este tipo de aprendizaje en problemas de 
clasificación constituye el voto conjunto, en donde los diferentes clasificadores votan sobre la clase a la que 
pertenece el elemento y, al final, se asigna la etiqueta que más votos obtuvo.

\subsection{Métodos de optimización}

El objetivo del AA es encontrar los extremos de una función de costo, este proceso es una tarea 
desafiante ya que la gran mayoría de estas funciones no son convexas y, por lo tanto, no existe un algoritmo
que asegure la convergencia hacia un extremo global. Para resolver este problema existen múltiples heurísticas,
la más usada es el descenso por gradiente. La idea básica consiste 
en el cálculo del vector gradiente de la función de error $f$ con respecto a los parámetros del modelo $x$ y, una vez se 
tiene dicho vector, se evalúa en la asignación actual de los parámetros $x_i$ y se realiza un corrimiento de este punto 
% TODO PREGUNTAR dice sustituir por {dirección} contra. No debe ser así no? 
en contra del gradiente para disminuir el error (Eq. \ref{eq:gradien_descent}).

\begin{equation}
	x_{i+1} = x_i - \alpha \nabla f(x_i)\label{eq:gradien_descent}
\end{equation}

En la ecuación \ref{eq:gradien_descent} anterior, $\alpha$ es la tasa de aprendizaje (\emph{learning rate}),
que cuantifica cuánto se toma del vector de gradiente para actualizar los parámetros, esto 
se puede ver como el aprendizaje del modelo.

Variantes eficientes de este algoritmo para el entrenamiento de modelos de AA han sido 
creadas, las variaciones se encuentran principalmente en la selección de $\alpha$ en cada paso y la 
selección de los conjuntos de datos con que se estimará el gradiente. Entre las técnicas utilizadas se 
encuentran Descenso por Gradiente Estocástico, variaciones de tasa de aprendizaje dinámica con 
sus diferentes variantes (exponencial, polinómica), RMSProp [\cite{tieleman2012rmsp}] 
y Adam [\cite{kingma2014adam}].

\section{Preliminares de Extracción de Argumentos}

Varias investigaciones han dado respuesta a los problemas asociados a EA, mostrando
una variedad en enfoques y métodos.

% TODO Cite problem, citar como texto
% [\autocite{palau2009argumentation}] 
% [\parencite{palau2009argumentation}] 
% [\citet{palau2009argumentation}] 
% [\citep{palau2009argumentation}] 
En \textcite{palau2009argumentation} se propone
el uso de modelos estadísticos como \emph{Naive Bayes} (NB) y \emph{Support Vector Machine} (SVM) 
para la clasificación de 
oraciones en argumentativas o no y en su rol argumentativo en caso de que sea argumentativa. En este
se asume que las componentes argumentativas son oraciones completas. Para la predicción de relaciones
se usa un enfoque basado en reglas con la creación de una Gramática Libre de Contexto. Las representaciones
de las oraciones consisten en atributos creados a mano, dado el conocimiento experto sobre la argumentación
en el tema tratado, elementos como adverbios, verbos, signos de puntuación, palabras clave, estadísticas del texto
(tamaño de oración, distancia media de palabras) son usados para la extracción y clasificación de las UDAs, además,
se usan también como base en la creación de las reglas de la gramática para la extracción de relaciones.

% TODO Cite problem, citar como texto
% [\autocite{goudas2015argument}] 
% [\parencite{goudas2015argument}] 
% [\citet{goudas2015argument}] 
% [\citep{goudas2015argument}] 
\textcite{goudas2015argument} al igual que \textcite{palau2009argumentation} clasifica a las oraciones como
argumentativas o no, mediante diferentes clasificadores como NB, \emph{Random Forest}, Regresión
Logística y SVM. Sin embargo, \textcite{goudas2015argument} aumenta la grandularidad de la segmentación al permitir
la extracción de los segmentos que contienen la carga argumentativa de dentro de las oraciones previamente clasificadas
como tal, esto se realiza mediante la extracción de etiquetas BIO de las oraciones con el uso de un 
CRF. La predicción de las relaciones es modelado como un problema de clasificación
% TODO PREGUNTAR dice eliminar {en}, pero no le veo el sentido a eso.
usando SVM para clasificar pares de UDAs en relacionados o no. Atributos creados a mano 
son usados en la extracción de UDAs; entre estos están posición de la oración en el texto, cantidad de verbos, comas, adverbios,
palabras, entidades en la oración, también se emplean listas que guardan entidades relacionadas con el dominio 
específico y palabras clave indicadoras de frases argumentativas. 

% TODO Cite problem, citar como texto
% [\autocite{stab2017parsing}] 
% [\parencite{stab2017parsing}] 
% [\citet{stab2017parsing}] 
% [\citep{stab2017parsing}] 
\textcite{stab2017parsing} proponen un mecanismo de segmentación basado en CRF. La clasificación
y predicción de relaciones se modela conjuntamente con dos clasificadores SVM y un problema
de Optimización Lineal Entero que encuentra la mejor estructura y asegura una disposición arbórea. En la segmentación
de las UDAs, se extraen por cada token su posición en el texto, si precede o sucede a un signo de puntuación, su parte de
la oración, la probabilidad de que sea el comienzo de una UDA dado sus tokens anteriores, entre otros. Para la extracción
y clasificación de relaciones se proponen otros conjuntos de atributos como la cantidad de sustantivos comunes entre
las componentes fuente y el objetivo, la presencia de indicadores argumentativos, representaciones vectoriales de tokens,
entre otros.

% TODO Cite problem, citar como texto
% [\autocite{eger2017neural}] 
% [\parencite{eger2017neural}] 
% [\citet{eger2017neural}] 
% [\citep{eger2017neural}] 
En \textcite{eger2017neural} trabajan el problema de EA como uno \emph{end-to-end}. 
Para esto presentaron varias propuestas, entre ellas se encontraba
modelar el problema como uno de secuencia a secuencia, usando RNN como 
LSTM en versiones bidireccionales capturando información desde ambos lados de la secuencia.
Para la representación de las palabras se extrajo información morfológica de las palabras mediante 
la aplicación de una CNN a los caracteres de estas,
al final, realizan la clasificación de la secuencia con un CRF. 
Realizaron experimentos al modelar el problema como uno de \emph{Dependency Parsing} [\cite{kiperwasser2016simple}]. Este problema
consiste en construir un árbol de dependencia que codifique las estructuras argumentativas. En este 
se tiene que decidir entre varias opciones (\emph{shift}, \emph{reduce}) en dependencia del contenido de la pila y del \emph{buffer}
para la confección del árbol.
El problema fue modelado también como un problema de reconocimiento de entidades nombradas, en donde las entidades son las UDAs.

% TODO Cite problem, citar como texto
% [\autocite{dykes2020reconstructing}] 
% [\parencite{dykes2020reconstructing}] 
% [\citet{dykes2020reconstructing}] 
% [\citep{dykes2020reconstructing}]
En \textcite{dykes2020reconstructing} se proponen métodos basados en reglas para la extracción de argumentos sobre
textos en Twitter. Estos métodos se centran en la confección de reglas basadas en anotaciones lingüísticas como
partes de la oración y lemas de palabras. La recuperación está basada en los esquemas argumentativos comunes presentes
en los textos. Dada las reglas creadas y el tipo
de datos con que se trabaja, o sea, cadenas de texto pequeñas; estos algoritmos tienden a tener una alta precisión aunque 
bajo recobrado, esto no es un gran problema en conjuntos de datos grandes, pero en conjuntos de menor tamaño o estructura 
más compleja pierden efectividad.

% TODO Cite problem, citar como texto
% [\autocite{galassi2021deep}] 
% [\parencite{galassi2021deep}] 
% [\citet{galassi2021deep}] 
% [\citep{galassi2021deep}]
\textcite{galassi2021deep} propone el uso de redes residuales y mecanismos de atención
para la creación de un modelo que, conjuntamente, clasifica el tipo de UDA y la relación existente entre estas.
Este trabajo define el concepto de distancia argumentativa, añadiéndolo como característica y asume que las UDAs ya fueron 
extraídas. En este caso, además de la distancia argumentativa, las secuencias son representadas 
vectorialmente con GloVe.

En resumen, se contemplan disímiles enfoques al problema de EA desde una perspectiva enmarcada en modelos 
simbólicos, estadísticos y neuronales en versiones tanto secuenciales como \emph{end-to-end}. 
Cada uno de estos modelos presentan sus ventajas y desventajas a la hora de construirlos, 
extenderlos y comprender su funcionamiento. En modelos simbólicos se presenta una alta
precisión en dominios específicos debido a que se construyen teniendo en cuenta reglas específicas a un
contexto dado. Estos modelos son poco escalables y difíciles de mantener ya que sus reglas son construídas
a mano y dicho proceso requiere de conocimiento experto y tiempo. Los modelos estadísticos 
se caracterizan por usar conjuntos de atributos creados a mano, dichos atributos son difíciles
de encontrar, calcular y pueden no poseer relevancia en otros contextos diferentes a los que fueron creados,
además, la necesidad de conocimiento experto es necesaria para su confección. Los modelos neuronales poseen
una mayor adaptabilidad, en estos la entrada puede ser codificada en una representación que es aprendida por
el mismo algoritmo, permitiendo su uso en esquemas argumentativos con características diferentes. Los modelos simbólicos y 
estadísticos poseen la ventaja de poder explicar el porqué de los resultados devueltos cosa que se vuelve casi
imposible en modelos neuronales.

Dado que la EA es un proceso en el cual se necesita pasar por varias tareas, estas deben de ser completadas
de alguna forma. Una manera de completarlas es hacerla una a la vez, independiente una de otra y pasándole
la salida de etapas anteriores a las etapas siguientes. Esta manera secuencial de realizar las 
tareas es bastante simple y ayuda a la creación de modelos simples y con tareas bien definidas, aunque trae consigo 
la propagación de los errores a través del proceso y el no aprovechamiento de las interrelaciones entre variables 
computadas de procesos anteriores. También requiere de la construcción, entrenamiento y evaluación de varios modelos.
En cambio un enfoque \emph{end-to-end} poseen la habilidad de modelar el problema 
desde su inicio hasta su final de manera conjunta, mediante \emph{Multi-Task Learning} (MTL) se modelan
las tareas de manera conjunta creando un solo modelo complejo con una propagación de error menor.

\section{Proyección de corpus}

La EA no presenta una gran cantidad de datos anotados con los cuales se pueda realizar 
un entrenamiento, además de esto la gran mayoría de corpus existentes se encuentran en lenguajes como inglés o alemán,
haciendo difícil el desarrollo de esta rama en otros lenguajes.
La escasez de estos datos es, en gran parte, debida al elevado costo monetario, de tiempo y de recursos humanos que se utiliza
en su creación. En orden de poder desarrollar la EA en otros lenguajes, como el español, se han investigado diferentes vertientes
para la construcción de conjuntos de datos en estos lenguajes de pocos recursos, a partir de los conjuntos de datos ya 
existentes.

La proyección de etiquetas consiste en un algoritmo en donde se 
transfieren las etiquetas de un corpus anotado a nivel de tokens en un lenguaje origen hacia su traducción en un
% TODO Cite problem, citar como texto
% [\autocite{eger2018cross}] 
% [\parencite{eger2018cross}] 
% [\citet{eger2018cross}] 
% [\citep{eger2018cross}] 
lenguaje objetivo. En \textcite{eger2018cross} se propone un algoritmo de proyección dadas las alineaciones de 
palabras. El proceso se divide en varias partes:

\begin{enumerate}
	\item Traducción de oraciones.
	\item Alineación de palabras.
	\item Proyección de etiquetas.
\end{enumerate}

\subsection{Traducción de oraciones}

La Traducción Automática consiste en el proceso de usar inteligencia artificial para
traducir texto de un lenguaje fuente a un lenguaje objetivo sin la intervención humana.
En la actualidad, este campo ha dado un gran paso pasando de modelos estadísticos a modelos
neuronales obteniendo traducciones de una alta calidad sin variar significativamente de la humana, 
condición necesaria para una buena proyección [\cite{eger2018cross}].

Este primer paso de la proyección de corpus consiste en traducir todas las oraciones existentes 
en el conjunto de datos hacia el lenguaje objetivo. 
% TODO Cite problem, citar como texto
% [\autocite{stab2017parsing}] 
% [\parencite{stab2017parsing}] 
% [\citet{stab2017parsing}] 
% [\citep{stab2017parsing}] 
A continuación se muestra una oración en inglés y su traducción al español\footnote{Extraído del corpus de \textcite{stab2017parsing}.}:

\begin{adjustwidth}{25pt}{25pt}
	Firstly , people normally have lots of things to do . \\
	En primer lugar , la gente normalmente tiene muchas cosas que hacer .
\end{adjustwidth}

\subsection{Alineación de palabras}

La alineación de palabras consiste en encontrar las palabras generadas en el lenguaje objetivo por las 
palabras en el lenguaje fuente.
Algoritmos basado en modelos bayesianos, como FastAlign [\cite{dyer2013fastalign}], 
y Cadenas de Markov-Monte Carlo, como EFEMARAL [\cite{ostling2016efficient}] se ubican entre
las primeras herramientas para la solución del problema. 
Modelos más recientes se han enfocado en explotar las representaciones
vectoriales de palabras y el uso de métodos de atención para la extracción de las
alineaciones [\cite{dou2021word}]. Algunas consideraciones sobre el proceso: las relaciones 
formadas entre palabras pueden ser de tipo muchos a muchos, además de no tener el mismo orden de la 
oración inicial o incluso no estar relacionadas directamente con una palabra en la oración objetivo.
Estas consideraciones dan una medida de la dificultad de la tarea en cuestión.
En el ejemplo siguiente se observa el resultado de las herramientas de alineación, en este 
las palabras en el idioma de origen (inglés) están anotadas con su posición en la oración y 
las palabras en el idioma objetivo (español) están anotadas con la posición de la palabra que 
la originó en el idioma origen:

\begin{adjustwidth}{25pt}{25pt}
	Firstly$_0$ ,$_1$ people$_2$ normally$_3$ have$_4$ lots$_5$ of$_6$ things$_7$ to$_8$ do$_9$ .$_{10}$ \\
	En primer$_0$ lugar$_0$ ,$_1$ la gente$_2$ normalmente$_3$ tiene$_4$ muchas$_5$ cosas$_7$ que$_8$ hacer$_9$ .$_{10}$
\end{adjustwidth}

\subsection{Proyección de etiquetas}

La proyección de etiquetas consiste en transportar las etiquetas de las palabras en la secuencia origen
hacia las palabras de la secuencia destino tomando como datos las alineaciones entre estas. En [\cite{yarowsky2001inducing}]
se trata el problema de proyección de frases nominales, estas frases tienen como característica que son resistentes
a ser divididas en caso de ser traducidas, y aunque evidencian 
cambios en el orden de las palabras, mantienen la misma ventana; dicha propiedad se cumple para las UDAs también.
La proyección de UDAs es más simple en dado
que solamente se tiene en cuenta la ventana y las etiquetas en estas son constantes, no pasa con la proyección en
frases nominales, las cuales pueden cambiar dentro de una ventana, por lo que algoritmos más simples existen
para esta tarea [\cite{eger2018cross}]. En el ejemplo de proyección están anotadas las etiquetas originales en formato BIO
de las palabras de la oración en el lenguaje origen (inglés) y se muestra el
resultado de proyectar estas al lenguaje objetivo utilizando los resultados de la alineación de palabras:

\begin{adjustwidth}{25pt}{25pt}
	Firstly$_O$ ,$_O$ people$_B$ normally$_I$ have$_I$ lots$_I$ of$_I$ things$_I$ to$_I$ do$_I$ .$_O$ \\
	En$_O$ primer$_O$ lugar$_O$ ,$_O$ la$_O$ gente$_B$ normalmente$_I$ tiene$_I$ muchas$_I$ cosas$_I$ que$_I$ hacer$_I$ .$_O$
\end{adjustwidth}

\chapter{Propuesta}\label{chapter:proposal}

\begin{itemize}

\item Modelo del fenómeno
\item Diseño conceptual
\begin{itemize}

\item Entrada $\rightarrow$ Procesamiento $\rightarrow$ Salida
\item Cómo se hace cada etapa

\end{itemize}

\end{itemize}
\chapter{Detalles de Implementación y Experimentos}\label{chapter:implementation}


\begin{itemize}

    \item No interesa el cómo se hizo
    \item Mostrar resultados
    \begin{itemize}

        \item Puede ser con estadísticas
        \item Por casos de uso

    \end{itemize}
    \item Sección de metacódigo en caso de ser relevante

\end{itemize}


\backmatter

\begin{conclusions}

% TODO Como se cumplieron los objetivos de la introduccion y para que sirve lo que hice

En el trabajo se implement'o un algoritmo con el cual se hace posible la extracción de las estructuras
argumentativas de textos 


\end{conclusions}

\begin{recomendations}
    Recomendaciones

    \begin{enumerate}
        \item Usar metodos de Graph Neural Networks para modelar las relaciones entre las UDA. (Actualmente las relaciones extraídas son independientes)
        \item Ajustar (Tune) awesome-align con inglés-español.
        \item Usar otros embeddings para la representación de palabras (BERT, SciBERT, RoBERTa).
        \item Desplegar un servidor Brat para socializar los resultados y permitir que los investigadores perfeccionen las anotaciones.
        \item Validación externa con lingüistas que permitan generalizar el uso del modelo.
    \end{enumerate}
\end{recomendations}

\printbibliography[heading=bibintoc]

%\begin{thebibliography}{99}
%
%\bibitem{texbook}
%Donald E. Knuth (1986) \emph{The \TeX{} Book}, Addison-Wesley Professional.
%
%
%\end{thebibliography}
%\begin{document}
%\bibliography{Bibliography}
%
%\end{document}

\end{document}
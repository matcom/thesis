\begin{opinion}
La lingüística computacional es una rama multidisciplinaria que procesa grandes 
cantidades de textos escritos. La tesis presentada por Luis Ernesto Ibarra Vázquez 
propone un algoritmo basado en modelos de aprendizaje automático para la segmentación 
y clasificación de enunciados argumentativos en textos de la sección “Cartas a la Dirección” 
del periódico \emph{Granma}.

La tesis se ubica en el marco de un proyecto \emph{Dinámicas sociales, políticas y económicas en el 
discurso público en Cuba de principio del siglo XXI: estudios de CORESPUC}, asociado al Programa 
Nacional de Ciencia y Técnica “Las Ciencias Sociales y las humanidades. Desafíos ante la estrategia 
de desarrollo de la sociedad cubana”, Código PN223LH011-011, Ministerio de Ciencia, Tecnología y 
Medio Ambiente (CITMA), Cuba, 2021-2023.

Por ello, Luis Ernesto tuvo que estudiar la materia referida, que no está incluida en el currículo 
de la carrera y trabajó mostrando creatividad, disciplina, entrega y rigor. Deseo destacar su 
profundización en el estudio computacional de la argumentación y la proposición de soluciones de 
software útiles para el análisis de enunciados argumentativos en español.

La investigación realizada por el estudiante incluyó diversos corpus en idioma inglés con 
diferentes anotaciones de la argumentación que se usaron para el entrenamiento de modelos de 
aprendizaje automático proyectivos para clasificar los enunciados argumentativos en español, a 
partir del conocimiento de la anotación en idioma inglés. Se compararon varios modelos y el mejor 
fue utilizado para la clasificación de enunciados de las Cartas a la Dirección del periódico Granma. 
Este trabajo es un primer paso positivo en el estudio automático de la argumentación de nuestra variante 
del español y llevará una posterior revisión por lingüistas. Dicho corpus anotado será importante para la 
realización de análisis sociolingüísticos del corpus donde la argumentación es una variable de interés para 
el proyecto CORESPUC.

Durante el desarrollo del trabajo, Luis Ernesto demostró habilidades para el trabajo con la bibliografía y 
creatividad para proponer soluciones a problemas de implementación, entre otras competencias de programación 
en el lenguaje Python y sus diversos \emph{frameworks}. Así, se logró cumplir el objetivo de esta tesis.

Por tanto, considero que al estudiante Luis Ernesto Ibarra Vázquez debe otorgársele la máxima calificación 
(5 puntos, Excelente), y estoy seguro que en el futuro Luis Ernesto se desempeñará como un excelente profesional 
de la Ciencia de la Computación.

\begin{flushright}
\includegraphics[scale=0.2]{Graphics/firma.jpg}\\
MSc. Damian Valdés Santiago\\
21 de noviembre de 2022    
\end{flushright}

\end{opinion}